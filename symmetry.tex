% SEC 7.1
\section[Diag. of Sym. Matrices]{Diagonalization of Symmetric Matrices}
\name[1.5in]

\begin{boxdef}
	A \textbf{symmetric matrix} is a matrix $A$ such that $A^T=A$.
\end{boxdef}
\vspace{-1em}
\begin{boxthm}
	\textbf{Theorem 7.1.} \\
	If $A$ is symmetric, then any two eigenvectors from different eigenspaces are orthogonal.
\end{boxthm}


\begin{exercise} % 7.1.19 Custom
	\begin{enumerate}[(a)]
		\item Fill in the following matrix so that it is symmetric:
%		$$ A = \begin{bmatrix}2&-4&8\\-4&8&4\\8&4&2\end{bmatrix} $$	
		\begingroup
		\renewcommand{\arraystretch}{1.5}
		$$ A = \begin{bmatrix}2&&  &&8\\-4&&8&&4\\ && &&2\end{bmatrix} $$
		\endgroup
		\item The vectors $\vect{v}_1=\begin{bmatrix}-1\\2\\0\end{bmatrix}$, $\vect{v}_2=\begin{bmatrix}1\\0\\1\end{bmatrix}$, and $\vect{v}_3=\begin{bmatrix}2\\1\\-2\end{bmatrix}$ are eigenvectors of the symmetric matrix $A$ above. The eigenvalue for $\vect{v}_1$ and $\vect{v}_2$ is $\lambda=10$, and the eigenvalue for $\vect{v}_3$ is $\lambda=-8$. What does Theorem 7.1 tell you about $\vect{v}_1\cdot\vect{v}_2$, $\vect{v}_2\cdot\vect{v}_3$, and $\vect{v}_1\cdot\vect{v}_3$, if anything?
	\end{enumerate}
\end{exercise}
\vspace{2in}


\begin{boxdef}
	An \textbf{orthogonal matrix} (section 6.2) is a square matrix $A$ with orthonormal columns (so $A^{-1}=A^T$). \par
	A matrix $A$ is \textbf{orthogonally diagonalizable} if there are an orthogonal matrix $P$ and a diagonal matrix $D$ such that $A=PDP^{-1}=PDP^T$.
\end{boxdef}
\vspace{-1em}
\begin{boxthm}
	\textbf{Theorem 7.2.} \\
	An $n\times n$ matrix $A$ is orthogonally diagonalizable if, and only if, $A$ is a symmetric matrix.
\end{boxthm}


\begin{exercise} % 7.1.9 Custom
	\begin{enumerate}[(a)]
		\item Is $B$ an orthogonal matrix? Why or why not? \par
		$ B = \begin{bmatrix}2&-3\\3&2\end{bmatrix} $
		\item Give an example of a $3\times 3$ matrix that is orthogonally diagonalizable.
	\end{enumerate}
\end{exercise}
\vfill


\newpage


\begin{boxme}
	\textbf{Steps to Orthogonally Diagonalize an $\boldsymbol{n\times n}$ Symmetric Matrix}
	\begin{enumerate}[(i)]\itemsep=0em
		\item Find the eigenvalues of $A$ 
		\item Find $n$ linearly independent eigenvectors for $A$, orthogonalize the set (if necessary), and normalize
		\item Construct $P$ from the orthonormalized eigenvectors you obtain in step (ii)
		\item Construct $D$ by placing the corresponding eigenvectors along the diagonal of $D$
	\end{enumerate}
\end{boxme}


\begin{exercise} % 7.1.15 Custom
	A matrix and two eigenvectors are given below. Orthogonally diagonalize the matrix (you only need to determine $P$ and $D$). Steps (i) and (ii) should go quickly since you already have 2 eigenvectors. \par
	$ A = \begin{bmatrix}13&4\\4&7\end{bmatrix}, 
	\vect{v}_1 = \begin{bmatrix}-1\\2\end{bmatrix}, 
	\vect{v}_2 = \begin{bmatrix}2\\1\end{bmatrix} $
%	\begin{align*}
%	A &= \begin{bmatrix}13&4\\4&7\end{bmatrix} &
%	\vect{v}_1 &= \begin{bmatrix}-1\\2\end{bmatrix}, 
%	\vect{v}_2 = \begin{bmatrix}2\\1\end{bmatrix}
%	\end{align*}
\end{exercise}
\vfill


\begin{boxdef}
	The set of eigenvalues of a matrix $A$ is called the \textbf{spectrum} of $A$. A \textbf{spectral decomposition} for $A$ can be obtained from the columns of $P$ and eigenvalues from $D$ in the orthogonal diagonalization of $A$:
	\begin{align*}
	A &= PDP^T
	= \begin{bmatrix}\vect{u}_1&\cdots&\vect{u}_n\end{bmatrix}
	\begin{bmatrix}\lambda_1&&0\\&\ddots&\\0&&\lambda_n\end{bmatrix}
	\begin{bmatrix}[c]\vect{u}_1^T\\ \vdots\\ \vect{u}_n^T\end{bmatrix}
	= \begin{bmatrix}\lambda_1\vect{u}_1&\cdots&\lambda_n\vect{u}_n\end{bmatrix}
	\begin{bmatrix}[c]\vect{u}_1^T\\ \vdots\\ \vect{u}_n^T\end{bmatrix}
	\end{align*}
	This last expression can be rewritten as a spectral decomposition:
	$A = \lambda_1\vect{u}_1\vect{u}_1^T + \lambda_2\vect{u}_2\vect{u}_2^T + \cdots + \lambda_n\vect{u}_n\vect{u}_n^T.$ This is a sum of matrices each of which relies on just one column of $P$ and its corresponding eigenvalue.
\end{boxdef}


\begin{exercise} % 7.1.15 Custom
	Find a spectral decomposition for the matrix $A$ using the given orthogonal diagonalization, $A=PDP^T$.
	\begin{align*}
	A &= \begin{bmatrix}2&9\\9&2\end{bmatrix} &
	PDP^T &= 
	\begin{bmatrix}-1/\sqrt{2}&1/\sqrt{2}\\1/\sqrt{2}&1/\sqrt{2}\end{bmatrix}
	\begin{bmatrix}-7&0\\0&11\end{bmatrix}
	\begin{bmatrix}-1/\sqrt{2}&1/\sqrt{2}\\1/\sqrt{2}&1/\sqrt{2}\end{bmatrix}
	\end{align*}
	\begin{enumerate}[(a)]
		\item Compute $\vect{u}_1\vect{u}_1^T$ and $\vect{u}_2\vect{u}_2^T$, where $\vect{u}_1$ and $\vect{u}_2$ are the first and second columns of $P$, respectively.
		\vspace{1in}
		\item Write a spectral decomposition for $A$.
		\vspace{3em}
	\end{enumerate}
\end{exercise}


%\newpage
%
%
%% SEC 7.2
%\section{Quadratic Forms}
%\name
%
%\begin{exercise} % 7.2.
%	Text here.
%\end{exercise}
%\vfill
%
%
%\begin{exercise} % 7.2.
%	Text here.
%\end{exercise}
%\vfill
%
%
%\newpage
%
%
%\begin{exercise} % 7.2.
%	Text here.
%\end{exercise}
%\vfill
%
%
%\begin{exercise} % 7.2.
%	Text here.
%\end{exercise}
%\vfill
%
%
%\newpage
%
%
%\setcounter{section}{3}
%\section{The Singular Value Decomposition}
%\name
%
%\begin{exercise} % 7.4.
%	Text here.
%\end{exercise}
%\vfill
%
%
%\begin{exercise} % 7.4.
%	Text here.
%\end{exercise}
%\vfill
%
%
%\newpage
%
%
%\begin{exercise} % 7.4.
%	Text here.
%\end{exercise}
%\vfill
%
%
%\begin{exercise} % 7.4.
%	Text here.
%\end{exercise}
%\vfill
