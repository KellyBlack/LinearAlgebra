\documentclass[svgnames,table]{beamer}
%\documentclass[svgnames,table,handout,aspectratio=129]{beamer}
\usepackage{hhline}
\usepackage{etoolbox}
\usepackage{tikz}
\usepackage{mathtools}
\usepackage{amssymb}
%\usepackage{/usr/lib64/R/share/texmf/Sweave}
\usepackage{polynom}
\usepackage{qrcode}


%\input{latexdefinitions}
\definecolor{georgiaRed}{RGB}{100,0,00}
\definecolor{mediumGray}{gray}{0.6}



\usetheme{Frankfurt}%
%\usetheme{Warsaw}%
%\useoutertheme{smoothbars}


%\usecolortheme{seagull}
\usecolortheme{beaver}
\logo{\includegraphics[height=.125in]{ugaLogo}}

% Note that the colour definitions are given in the latexDefinitions
% file.
\setbeamercolor{palette primary}{fg=georgiaRed,bg=white}
\setbeamercolor{palette secondary}{fg=georgiaRed,bg=white}
\setbeamercolor{palette tertiary}{fg=georgiaRed,bg=white}
\setbeamercolor{palette quaternary}{bg=mediumGray,fg=black}
\setbeamercolor{block title}{fg=black,bg=black!15}
\setbeamercolor{block body}{fg=black,bg=black!10}
\setbeamercolor{titlelike}{bg=georgiaRed,fg=white} % parent=palette quaternary}

% Define the variable to determine whether or not the clicker quizzes
% are visible in the resulting output.
\newtoggle{clicker}
\toggletrue{clicker}
%\togglefalse{clicker}


% To display a lecture uncomment out the "includeonly" line below to
% match the name of the file. You do not have to do anything with the
% lecture line below and can leave it commented out. It is in place
% because at one time we had multiple lectures within a file, but that
% has been changed.



\mode<presentation>{
  \setbeamercovered{invisible}
  \setbeameroption{hide notes}
}

\mode<handout>{
 
  \usepackage{pgfpages}
  %\pgfpagesuselayout{4 on 1}[letterpaper, border shrink=5mm]
  \pgfpagesuselayout{resize to}[letterpaper,border shrink=5mm]
  \setbeameroption{show notes}


  %\pgfpagesphysicalpageoptions{logical pages=2,physical
  %height=\pgfpageoptionheight,physical width=\pgfpageoptionwidth}
  % Set up the pages for notes.
  % This idea and some code came from
  % http://www.guidodiepen.nl/2009/07/creating-latex-beamer-handouts-with-notes/



  \pgfpagesdeclarelayout{3 on 1 with notes} {
    \edef\pgfpageoptionheight{11in} %\the\paperheight}
    \edef\pgfpageoptionwidth{8.5in} %\the\paperwidth}
    \edef\pgfpageoptionborder{0pt}
  }

  {

	\AtBeginDocument{
      \newbox\notesbox
      \setbox\notesbox=\vbox{
        \hsize=\paperwidth
        \vskip-2.5cm\hskip-5cm\vbox{
          \textcolor{light-gray}{\hrule width 12.6cm\vskip0.5cm}
          \textcolor{light-gray}{\hrule width 12.6cm\vskip0.5cm}
          \textcolor{light-gray}{\hrule width 12.6cm\vskip0.5cm}
          \textcolor{light-gray}{\hrule width 12.6cm\vskip0.5cm}
          \textcolor{light-gray}{\hrule width 12.6cm\vskip0.5cm}
          \textcolor{light-gray}{\hrule width 12.6cm\vskip0.5cm}
          \textcolor{light-gray}{\hrule width 12.6cm\vskip0.5cm}
          \textcolor{light-gray}{\hrule width 12.6cm\vskip0.5cm}
          \textcolor{light-gray}{\hrule width 12.6cm\vskip0.5cm}
          \textcolor{light-gray}{\hrule width 12.6cm\vskip0.5cm}
          \textcolor{light-gray}{\hrule width 12.6cm\vskip0.5cm}
          \textcolor{light-gray}{\hrule width 12.6cm\vskip0.5cm}
          \textcolor{light-gray}{\hrule width 12.6cm\vskip0.5cm}
          \textcolor{light-gray}{\hrule width 12.6cm\vskip0.5cm}
          \textcolor{light-gray}{\hrule width 12.6cm\vskip0.5cm}
          \textcolor{light-gray}{\hrule width 12.6cm\vskip0.5cm}
          \textcolor{light-gray}{\hrule width 12.6cm\vskip0.5cm}
          \textcolor{light-gray}{\hrule width 12.6cm\vskip0.5cm}
          \textcolor{light-gray}{\hrule width 12.6cm\vskip0.5cm}

          \vspace*{-9.75cm}
          \textcolor{light-gray}{\rule[-1.0cm]{1pt}{9.25cm}\hskip0.5cm}
          \textcolor{light-gray}{\rule[-1.0cm]{1pt}{9.25cm}\hskip0.5cm}
          \textcolor{light-gray}{\rule[-1.0cm]{1pt}{9.25cm}\hskip0.5cm}
          \textcolor{light-gray}{\rule[-1.0cm]{1pt}{9.25cm}\hskip0.5cm}
          \textcolor{light-gray}{\rule[-1.0cm]{1pt}{9.25cm}\hskip0.5cm}
          \textcolor{light-gray}{\rule[-1.0cm]{1pt}{9.25cm}\hskip0.5cm}
          \textcolor{light-gray}{\rule[-1.0cm]{1pt}{9.25cm}\hskip0.5cm}
          \textcolor{light-gray}{\rule[-1.0cm]{1pt}{9.25cm}\hskip0.5cm}
          \textcolor{light-gray}{\rule[-1.0cm]{1pt}{9.25cm}\hskip0.5cm}
          \textcolor{light-gray}{\rule[-1.0cm]{1pt}{9.25cm}\hskip0.5cm}
          \textcolor{light-gray}{\rule[-1.0cm]{1pt}{9.25cm}\hskip0.5cm}
          \textcolor{light-gray}{\rule[-1.0cm]{1pt}{9.25cm}\hskip0.5cm}
          \textcolor{light-gray}{\rule[-1.0cm]{1pt}{9.25cm}\hskip0.5cm}
          \textcolor{light-gray}{\rule[-1.0cm]{1pt}{9.25cm}\hskip0.5cm}
          \textcolor{light-gray}{\rule[-1.0cm]{1pt}{9.25cm}\hskip0.5cm}
          \textcolor{light-gray}{\rule[-1.0cm]{1pt}{9.25cm}\hskip0.5cm}
          \textcolor{light-gray}{\rule[-1.0cm]{1pt}{9.25cm}\hskip0.5cm}
          \textcolor{light-gray}{\rule[-1.0cm]{1pt}{9.25cm}\hskip0.5cm}
          \textcolor{light-gray}{\rule[-1.0cm]{1pt}{9.25cm}\hskip0.5cm}
          \textcolor{light-gray}{\rule[-1.0cm]{1pt}{9.25cm}\hskip0.5cm}

        }

      }

    \pgfpagesphysicalpageoptions
    {%
      logical pages=6,%
      physical height=\pgfpageoptionheight,%
      physical width=\pgfpageoptionwidth,%
      last logical shipout=3%
    }
    
    \pgfpageslogicalpageoptions{1}
    {%
      border shrink=\pgfpageoptionborder,%
      resized width=.5\pgfphysicalwidth,%
      resized height=.33\pgfphysicalheight,%
      center=\pgfpoint{.25\pgfphysicalwidth}{.82\pgfphysicalheight}%
    }%
    \pgfpageslogicalpageoptions{2}
    {%
      border shrink=\pgfpageoptionborder,%
      resized width=.5\pgfphysicalwidth,%
      resized height=.33\pgfphysicalheight,%
      center=\pgfpoint{.25\pgfphysicalwidth}{.47\pgfphysicalheight}%
    }%
    \pgfpageslogicalpageoptions{3}
    {%
      border shrink=\pgfpageoptionborder,%
      resized width=.5\pgfphysicalwidth,%
      resized height=.33\pgfphysicalheight,%
      center=\pgfpoint{.25\pgfphysicalwidth}{.17\pgfphysicalheight}%
    }%	
	\pgfpageslogicalpageoptions{4}
    {%
      border shrink=\pgfpageoptionborder,%
      resized width=.5\pgfphysicalwidth,%
      resized height=.33\pgfphysicalheight,%
      center=\pgfpoint{.85\pgfphysicalwidth}{.82\pgfphysicalheight},%
      copy from=4
    }%
    \pgfpageslogicalpageoptions{5}
    {%
      border shrink=\pgfpageoptionborder,%
      resized width=.5\pgfphysicalwidth,%
      resized height=.33\pgfphysicalheight,%
      center=\pgfpoint{.85\pgfphysicalwidth}{.47\pgfphysicalheight},%
      copy from=5
    }%
    \pgfpageslogicalpageoptions{6}
    {%
      border shrink=\pgfpageoptionborder,%
      resized width=.5\pgfphysicalwidth,%
      resized height=.33\pgfphysicalheight,%
      center=\pgfpoint{.85\pgfphysicalwidth}{.17\pgfphysicalheight},%
      copy from=6
    }%
    
      \pgfpagesshipoutlogicalpage{4}\copy\notesbox
      \pgfpagesshipoutlogicalpage{5}\copy\notesbox
      \pgfpagesshipoutlogicalpage{6}\copy\notesbox
    }
  }

  \pgfpagesuselayout{3 on 1 with notes}

}

\setbeamercolor{upper separation line head}{bg=red}
\setbeamercolor{headline}{bg=red}
\setbeamertemplate{headline}
{%
\begin{beamercolorbox}{section in head/foot}
\insertsectionnavigationhorizontal{.75\textwidth}{}{}
\hfill \insertpagenumber /\insertdocumentendpage
\end{beamercolorbox}%
}
\setbeamercolor{section number projected}{bg=red,fg=black}
\setbeamercolor{subsection number projected}{bg=red,fg=black}
%\setbeamercolor{frametitle}{bg=lightgray,fg=black}

\setbeamertemplate{itemize item}{\color{georgiaRed}$\blacklozenge$}
\setbeamertemplate{itemize subitem}{\color{georgiaRed}$\blacktriangleright$}

\begin{document}



\author{\textsc{K. Black$^{a}$, T. Alli$^{a}$}}
\institute{$^a$Department of Mathematics, University of Georgia, GA}
\subject{Linear Algebra}
\keywords{Linear Transformation, Vectors, Matrices, Linear Algebra}

%\lecture{Partial Fractions}{partial-fractions}
%\section{Rational Functions}

\title{Introduction To Linear Algebra}
\subtitle{Exploring Linear Transformations}

\date{May 2020} % {\today}

\begin{frame}
  \titlepage
\end{frame}

\begin{frame}{Outline}
  \tableofcontents
\end{frame}


\section{Introduction}

\begin{frame}{Welcome!}

  Why Linear Algebra?
  \begin{itemize}
  \item Used in all science/tech/math fields
  \item Incredibly useful tools
  \item Common practice to transform a problem into a linear algebra
    question
  \item Look at the world in terms of systems
  \end{itemize}

\end{frame}

\begin{frame}{Central Ideas}

  \begin{itemize}
  \item Linear Relationships
  \item Systems
  \item Transformation
    \begin{itemize}
    \item How to characterize a linear transformation?
    \item How to represent a linear transformation?
    \item How to perform calculations associated with linear transformations?
    \item Which linear transformations?
    \item How to break a transformation into parts? (Decomposition)
    \end{itemize}
  \end{itemize}

\end{frame}

\section{Systems}

\begin{frame}{Systems}

  Systems are everywhere!
  \begin{itemize}
  \item Multiple forces acting on multiple objects
  \item Circuits
  \item Chemical reactions
  \item Interactions between different species of animals \\
    (Cooperation, competition, predator/prey, ....)
  \item Many others!
  \end{itemize}

\end{frame}



\section{Representing A Vector}

\begin{frame}{Representing Vectors}
  \begin{description}
  \item[Engineering] 5N at an angle of $53.1^\circ$ from the
    horizontal.
    
  \item[Science] $3\vec{\imath}+4\vec{\jmath}$
    
  \item[Mathematics]
    $\left[
      \begin{array}{r}
        3 \\ 4
      \end{array}
    \right]$
  \item[Notation] $\textbf{x}$, $\vec{x}$, $\hat{x}$
  \end{description}
\end{frame}

\section{Topics}

\begin{frame}
  \frametitle{Topics}
  Linear Systems, Vectors, Planes
  \begin{columns}
    \column{0.5\linewidth}

    \begin{itemize}
    \item Matrices and Matrix Representations
    \item Operations
      \begin{itemize}
      \item Matrix Operations
      \item Row Operations
      \end{itemize}
    \item Inverse
    \item Decomposition
    \item Others....
    \end{itemize}

    \column{0.5\linewidth}

    \begin{itemize}
    \item Vector sums
    \item Matrix multiplication as a linear transformation
    \item Column space of a matrix
    \item Linear independence
    \item Linear Combinations
    \item Others...
    \end{itemize}

  \end{columns}
\end{frame}

\section{Example: Statics}

\begin{frame}
  \frametitle{Example: Statics}

  A box is held up in the air by two ropes. The rope on the left forms
  an angle of $\phi$ with the horizon, and the rope on the right forms
  an angle of $\theta$ with the horizon.  The box has a mass of
  30kg. What are the tensions in the rope?
  
\end{frame}


\begin{frame}
  \frametitle{Example: Statics}

  ~
  
\end{frame}

\begin{frame}
  \frametitle{Example: Statics}

  \center
  \includegraphics[width=7cm]{freeBodyDiagram}
  
\end{frame}

\begin{frame}
  \frametitle{Example: Statics}

  \center
  \includegraphics[width=7cm]{freeBodyDiagram}
  
\end{frame}

\end{document}
