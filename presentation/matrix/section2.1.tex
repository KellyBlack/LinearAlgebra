\documentclass[svgnames,table,,aspectratio=169]{beamer}
%\documentclass[svgnames,table,handout,aspectratio=129]{beamer}
\usepackage{hhline}
\usepackage{etoolbox}
\usepackage{tikz}
\usepackage{mathtools}
\usepackage{amssymb}
%\usepackage{/usr/lib64/R/share/texmf/Sweave}
\usepackage{polynom}
\usepackage{qrcode}


%\input{latexdefinitions}
\definecolor{georgiaRed}{RGB}{100,0,00}
\definecolor{mediumGray}{gray}{0.6}



\usetheme{Frankfurt}%
%\usetheme{Warsaw}%
%\useoutertheme{smoothbars}


%\usecolortheme{seagull}
\usecolortheme{beaver}
\logo{\includegraphics[height=.125in]{ugaLogo}}

% Note that the colour definitions are given in the latexDefinitions
% file.
\setbeamercolor{palette primary}{fg=georgiaRed,bg=white}
\setbeamercolor{palette secondary}{fg=georgiaRed,bg=white}
\setbeamercolor{palette tertiary}{fg=georgiaRed,bg=white}
\setbeamercolor{palette quaternary}{bg=mediumGray,fg=black}
\setbeamercolor{block title}{fg=black,bg=black!15}
\setbeamercolor{block body}{fg=black,bg=black!10}
\setbeamercolor{titlelike}{bg=georgiaRed,fg=white} % parent=palette quaternary}

% Define the variable to determine whether or not the clicker quizzes
% are visible in the resulting output.
\newtoggle{clicker}
\toggletrue{clicker}
%\togglefalse{clicker}


% To display a lecture uncomment out the "includeonly" line below to
% match the name of the file. You do not have to do anything with the
% lecture line below and can leave it commented out. It is in place
% because at one time we had multiple lectures within a file, but that
% has been changed.



\mode<presentation>{
  \setbeamercovered{invisible}
  \setbeameroption{hide notes}
}

\mode<handout>{
 
  \usepackage{pgfpages}
  %\pgfpagesuselayout{4 on 1}[letterpaper, border shrink=5mm]
  \pgfpagesuselayout{resize to}[letterpaper,border shrink=5mm]
  \setbeameroption{show notes}


  %\pgfpagesphysicalpageoptions{logical pages=2,physical
  %height=\pgfpageoptionheight,physical width=\pgfpageoptionwidth}
  % Set up the pages for notes.
  % This idea and some code came from
  % http://www.guidodiepen.nl/2009/07/creating-latex-beamer-handouts-with-notes/



  \pgfpagesdeclarelayout{3 on 1 with notes} {
    \edef\pgfpageoptionheight{11in} %\the\paperheight}
    \edef\pgfpageoptionwidth{8.5in} %\the\paperwidth}
    \edef\pgfpageoptionborder{0pt}
  }

  {

	\AtBeginDocument{
      \newbox\notesbox
      \setbox\notesbox=\vbox{
        \hsize=\paperwidth
        \vskip-2.5cm\hskip-5cm\vbox{
          \textcolor{light-gray}{\hrule width 12.6cm\vskip0.5cm}
          \textcolor{light-gray}{\hrule width 12.6cm\vskip0.5cm}
          \textcolor{light-gray}{\hrule width 12.6cm\vskip0.5cm}
          \textcolor{light-gray}{\hrule width 12.6cm\vskip0.5cm}
          \textcolor{light-gray}{\hrule width 12.6cm\vskip0.5cm}
          \textcolor{light-gray}{\hrule width 12.6cm\vskip0.5cm}
          \textcolor{light-gray}{\hrule width 12.6cm\vskip0.5cm}
          \textcolor{light-gray}{\hrule width 12.6cm\vskip0.5cm}
          \textcolor{light-gray}{\hrule width 12.6cm\vskip0.5cm}
          \textcolor{light-gray}{\hrule width 12.6cm\vskip0.5cm}
          \textcolor{light-gray}{\hrule width 12.6cm\vskip0.5cm}
          \textcolor{light-gray}{\hrule width 12.6cm\vskip0.5cm}
          \textcolor{light-gray}{\hrule width 12.6cm\vskip0.5cm}
          \textcolor{light-gray}{\hrule width 12.6cm\vskip0.5cm}
          \textcolor{light-gray}{\hrule width 12.6cm\vskip0.5cm}
          \textcolor{light-gray}{\hrule width 12.6cm\vskip0.5cm}
          \textcolor{light-gray}{\hrule width 12.6cm\vskip0.5cm}
          \textcolor{light-gray}{\hrule width 12.6cm\vskip0.5cm}
          \textcolor{light-gray}{\hrule width 12.6cm\vskip0.5cm}

          \vspace*{-9.75cm}
          \textcolor{light-gray}{\rule[-1.0cm]{1pt}{9.25cm}\hskip0.5cm}
          \textcolor{light-gray}{\rule[-1.0cm]{1pt}{9.25cm}\hskip0.5cm}
          \textcolor{light-gray}{\rule[-1.0cm]{1pt}{9.25cm}\hskip0.5cm}
          \textcolor{light-gray}{\rule[-1.0cm]{1pt}{9.25cm}\hskip0.5cm}
          \textcolor{light-gray}{\rule[-1.0cm]{1pt}{9.25cm}\hskip0.5cm}
          \textcolor{light-gray}{\rule[-1.0cm]{1pt}{9.25cm}\hskip0.5cm}
          \textcolor{light-gray}{\rule[-1.0cm]{1pt}{9.25cm}\hskip0.5cm}
          \textcolor{light-gray}{\rule[-1.0cm]{1pt}{9.25cm}\hskip0.5cm}
          \textcolor{light-gray}{\rule[-1.0cm]{1pt}{9.25cm}\hskip0.5cm}
          \textcolor{light-gray}{\rule[-1.0cm]{1pt}{9.25cm}\hskip0.5cm}
          \textcolor{light-gray}{\rule[-1.0cm]{1pt}{9.25cm}\hskip0.5cm}
          \textcolor{light-gray}{\rule[-1.0cm]{1pt}{9.25cm}\hskip0.5cm}
          \textcolor{light-gray}{\rule[-1.0cm]{1pt}{9.25cm}\hskip0.5cm}
          \textcolor{light-gray}{\rule[-1.0cm]{1pt}{9.25cm}\hskip0.5cm}
          \textcolor{light-gray}{\rule[-1.0cm]{1pt}{9.25cm}\hskip0.5cm}
          \textcolor{light-gray}{\rule[-1.0cm]{1pt}{9.25cm}\hskip0.5cm}
          \textcolor{light-gray}{\rule[-1.0cm]{1pt}{9.25cm}\hskip0.5cm}
          \textcolor{light-gray}{\rule[-1.0cm]{1pt}{9.25cm}\hskip0.5cm}
          \textcolor{light-gray}{\rule[-1.0cm]{1pt}{9.25cm}\hskip0.5cm}
          \textcolor{light-gray}{\rule[-1.0cm]{1pt}{9.25cm}\hskip0.5cm}

        }

      }

    \pgfpagesphysicalpageoptions
    {%
      logical pages=6,%
      physical height=\pgfpageoptionheight,%
      physical width=\pgfpageoptionwidth,%
      last logical shipout=3%
    }
    
    \pgfpageslogicalpageoptions{1}
    {%
      border shrink=\pgfpageoptionborder,%
      resized width=.5\pgfphysicalwidth,%
      resized height=.33\pgfphysicalheight,%
      center=\pgfpoint{.25\pgfphysicalwidth}{.82\pgfphysicalheight}%
    }%
    \pgfpageslogicalpageoptions{2}
    {%
      border shrink=\pgfpageoptionborder,%
      resized width=.5\pgfphysicalwidth,%
      resized height=.33\pgfphysicalheight,%
      center=\pgfpoint{.25\pgfphysicalwidth}{.47\pgfphysicalheight}%
    }%
    \pgfpageslogicalpageoptions{3}
    {%
      border shrink=\pgfpageoptionborder,%
      resized width=.5\pgfphysicalwidth,%
      resized height=.33\pgfphysicalheight,%
      center=\pgfpoint{.25\pgfphysicalwidth}{.17\pgfphysicalheight}%
    }%	
	\pgfpageslogicalpageoptions{4}
    {%
      border shrink=\pgfpageoptionborder,%
      resized width=.5\pgfphysicalwidth,%
      resized height=.33\pgfphysicalheight,%
      center=\pgfpoint{.85\pgfphysicalwidth}{.82\pgfphysicalheight},%
      copy from=4
    }%
    \pgfpageslogicalpageoptions{5}
    {%
      border shrink=\pgfpageoptionborder,%
      resized width=.5\pgfphysicalwidth,%
      resized height=.33\pgfphysicalheight,%
      center=\pgfpoint{.85\pgfphysicalwidth}{.47\pgfphysicalheight},%
      copy from=5
    }%
    \pgfpageslogicalpageoptions{6}
    {%
      border shrink=\pgfpageoptionborder,%
      resized width=.5\pgfphysicalwidth,%
      resized height=.33\pgfphysicalheight,%
      center=\pgfpoint{.85\pgfphysicalwidth}{.17\pgfphysicalheight},%
      copy from=6
    }%
    
      \pgfpagesshipoutlogicalpage{4}\copy\notesbox
      \pgfpagesshipoutlogicalpage{5}\copy\notesbox
      \pgfpagesshipoutlogicalpage{6}\copy\notesbox
    }
  }

  \pgfpagesuselayout{3 on 1 with notes}

}

\setbeamercolor{upper separation line head}{bg=red}
\setbeamercolor{headline}{bg=red}
\setbeamertemplate{headline}
{%
\begin{beamercolorbox}{section in head/foot}
\insertsectionnavigationhorizontal{.75\textwidth}{}{}
\hfill \insertpagenumber /\insertdocumentendpage
\end{beamercolorbox}%
}
\setbeamercolor{section number projected}{bg=red,fg=black}
\setbeamercolor{subsection number projected}{bg=red,fg=black}
%\setbeamercolor{frametitle}{bg=lightgray,fg=black}

\setbeamertemplate{itemize item}{\color{georgiaRed}$\blacklozenge$}
\setbeamertemplate{itemize subitem}{\color{georgiaRed}$\blacktriangleright$}

\newcommand{\dotfield}[2]{%
  \begin{tikzpicture}[y=0.25cm, x=0.25cm,font=\sffamily]
    \foreach \y in {0,...,#2} {
      \foreach \x in {0,...,#1} {
        \draw[fill=georgiaRed,opacity=0.1] (\x,\y)  circle [radius=0.03em];
      }
    }
  \end{tikzpicture}
}

\newcommand{\twoByTwo}[4]{
  \left[
    \begin{array}{rr}
      #1 & #2 \\
      #3 & #4 \\
    \end{array}
  \right]
}

\newcommand{\threeByThree}[9]{
  \left[
    \begin{array}{rrr}
      #1 & #2 & #3 \\
      #4 & #5 & #6 \\
      #7 & #8 & #9
    \end{array}
  \right]
}

\newcommand{\columnVector}[1]{%
  \left[
    \begin{array}{r}
    #1                           
    \end{array}
  \right]
}


\begin{document}



\author{\textsc{K. Black$^{a}$, T. Alli$^{a}$}}
\institute{$^a$Department of Mathematics, University of Georgia, GA}
\subject{Linear Algebra}
\keywords{Linear Transformation, Vectors, Matrices, Linear Algebra}

%\lecture{Partial Fractions}{partial-fractions}
%\section{Rational Functions}

\title{Section 2.1 - Matrix Algebra}
\subtitle{Operations on matrices}

\date{} % {\today}

\begin{frame}
  \titlepage
\end{frame}

\begin{frame}{Outline}
  \tableofcontents
\end{frame}


\section{Goals}

\begin{frame}{Goals}

  \begin{itemize}
  \item Perform basic matrix operations.
  \item Distinguish between scalar, vector, and matrix operations.
  \item Recognize order of operations.
  \item Make use of correct matrix notation.
  \item Perform operations using the transpose of a matrix.
  \end{itemize}

\end{frame}

\section{Basic Matrix Operations}

\begin{frame}{Matrix Addition}

  \begin{columns}
    \column{0.5\textwidth}
    \begin{eqnarray*}
      A & = & \twoByTwo{3}{-4}{2}{1} \\
      B & = & \twoByTwo{1}{2}{-5}{7} \\ [10pt]
      A+B & = & \twoByTwo{1+3}{-4+2}{2+(-5)}{1+7} \\ [10pt]
      4A  & = & \twoByTwo{4\cdot 3}{4\cdot (-4)}{4\cdot 2}{4\cdot 1}
    \end{eqnarray*}

    \column<2->{0.5\textwidth}
    \begin{itemize}
    \item $A+B=B+A$.
    \item $(A+B)+C=A+(B+C)$
    \item $r(A+B) = rA + rB$
    \item $(r+s)A = rA + sB$
    \item $r(sA)=(rs)A$
    \end{itemize}
  \end{columns}
\end{frame}

\begin{frame}{Matrix Multiplication}

  If $B$ is a matrix, and the columns of $B$ are
  \begin{eqnarray*}
    \vec{b}_1, ~ \vec{b}_2, ~ \vec{b}_3, ~ \ldots ~ , \vec{b}_n,
  \end{eqnarray*}
  then
  \begin{eqnarray*}
    B\vec{x} & = &  x_1 \vec{b}_1 + x_2 \vec{b}_2
                   + x_3 \vec{b}_3 +  \ldots +  x_n \vec{b}_n.
  \end{eqnarray*}

  Wait, if $B\vec{x}$ is a vector, then that means that we can
  multiply $B\vec{x}$ by another matrix! So
  \begin{eqnarray*}
    AB\vec{x}
  \end{eqnarray*}
  can be done, and it is defined if we multiply in the following order:
  \begin{eqnarray*}
    A\left(B\vec{x}\right).
  \end{eqnarray*}

  Question: Can we define multiplication of two matrices so that
  \begin{eqnarray*}
    \left(AB\right)\vec{x} & = & A\left(B\vec{x}\right).
  \end{eqnarray*}

  \uncover<2>{YES WE CAN!!!}
  
\end{frame}

\begin{frame}{Matrix Multiplication}

  Going back to the definition, we have that
  \begin{eqnarray*}
    B\vec{x} & = &  x_1 \vec{b}_1 + x_2 \vec{b}_2
                   + x_3 \vec{b}_3 +  \ldots +  x_n \vec{b}_n.
  \end{eqnarray*}

  So.....
  \begin{eqnarray*}
    A\left(B\vec{x}\right) & = & A \left(
                                 x_1 \vec{b}_1 + x_2 \vec{b}_2
                                 + x_3 \vec{b}_3 +  \ldots +  
                                 x_n \vec{b}_n
                                 \right).
  \end{eqnarray*}
  Now distributing the A we get
  \begin{eqnarray*}
    A\left(B\vec{x}\right) & = &  x_1 A\vec{b}_1 + x_2 A\vec{b}_2
                                 + x_3 A\vec{b}_3 +  \ldots +  
                                 x_n A\vec{b}_n.
  \end{eqnarray*}
  This implies that the $\jmath$\textsuperscript{th} column of the matrix
  $AB$ is $A\vec{b}_j$. 
\end{frame}

\begin{frame}{Matrix-Matrix Multiplication}

  We define matrix-matrix multiplication to be
  \begin{eqnarray*}
    AB & = & A \left[\vec{b}_1 ~ \vec{b}_2  ~ \vec{b}_3  ~ \ldots ~ \vec{b}_n\right], \\
       & = & \left[A\vec{b}_1  ~ A\vec{b}_2 ~ A\vec{b}_3 ~ \ldots ~ A\vec{b}_n\right]. \\
  \end{eqnarray*}
  
\end{frame}

\begin{frame}{Example}

  \begin{eqnarray*}
    \twoByTwo{3}{-4}{2}{1} \twoByTwo{1}{2}{-5}{7} & = & ?
  \end{eqnarray*}

  \dotfield{60}{18}
  
\end{frame}

\section{The Algebra of Matrix-Matrix Multiplication}

\begin{frame}{Matrix-Matrix Multiplication}

  \begin{itemize}
  \item $A(BC)=(AB)C$.
  \item $A(B+C)=AB+AC$.
  \item $(A+B)C=AC+BC$.
  \item<2-> $AB\neq BA$. 
  \end{itemize}
  
\end{frame}

\begin{frame}{Exponential Notation}

  We can define powers of a matrix:
  \begin{eqnarray*}
    A^2 & = & A\cdot A, \\
    A^3 & = & A\cdot A \cdot A, \\
    A^4 & = & A\cdot A  \cdot A  \cdot A, \\
    A^n & = & \underbrace{A  \cdot A  \cdot A  \cdot A \cdots A}_\text{$n$~times}
  \end{eqnarray*}
  
\end{frame}

\section{The Transpose}

\newcommand{\rowFour}[4]{
  #1 & #2 & #3 & #4
}

\newcommand{\threeRows}[3]{
  \left[
    \begin{array}{rrrr}
      #1 \\
      #2 \\
      #3 
    \end{array}
  \right]
}


\begin{frame}{The Transpose}

  Throughout the semester we are going to define some operations that
  seem to come out of nowhere. These operations pop up in many
  different situations and turn out to be useful. Like everything
  useful in mathematics we will define them using arcane notation and
  just get on with our lives\footnote{Sorry!}.

  An example of one of these operations is the transpose. Basically we
  will take the columns of a matrix and turn them on their sides so
  they become the rows of a new matrix.
  
\end{frame}

\begin{frame}{Examples Of The Transpose}

  \begin{eqnarray*}
    A & = & \threeByThree{7}{18}{-4}{33}{1}{20}{17}{-13}{2}, \\
    A^T & = & 
  \end{eqnarray*}
  
\end{frame}

\begin{frame}{Examples Of The Transpose}

  \begin{eqnarray*}
    A & = & \threeRows{%
            \rowFour{3}{-1}{5}{2}}{%
            \rowFour{10}{2}{1}{4}}{%
            \rowFour{8}{7}{6}{11}}, \\
    A^T & = & 
  \end{eqnarray*}
  
\end{frame}

\begin{frame}{Matrix Operations With The Transpose}

  \begin{itemize}
  \item $\left(A^T\right)^T = A$
  \item $\left(A+B\right)^T = A^T + B^T$
  \item $\left(rA\right)^T = rA^T$
  \item $\left(AB\right)^T = B^T A^T$
  \end{itemize}
  
\end{frame}

%\begin{frame}{Blank Page}
%  \dotfield{60}{24}
%\end{frame}



\end{document}
