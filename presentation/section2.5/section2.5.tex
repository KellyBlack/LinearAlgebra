\documentclass[svgnames,table,,aspectratio=169]{beamer}
%\documentclass[svgnames,table,handout,aspectratio=129]{beamer}
\usepackage{hhline}
\usepackage{etoolbox}
\usepackage{tikz}
\usepackage{mathtools}
\usepackage{amssymb}
%\usepackage{/usr/lib64/R/share/texmf/Sweave}
\usepackage{polynom}
\usepackage{qrcode}


%\input{latexdefinitions}
\definecolor{georgiaRed}{RGB}{100,0,00}
\definecolor{mediumGray}{gray}{0.6}



\usetheme{Frankfurt}%
%\usetheme{Warsaw}%
%\useoutertheme{smoothbars}


%\usecolortheme{seagull}
\usecolortheme{beaver}
\logo{\includegraphics[height=.125in]{ugaLogo}}

% Note that the colour definitions are given in the latexDefinitions
% file.
\setbeamercolor{palette primary}{fg=georgiaRed,bg=white}
\setbeamercolor{palette secondary}{fg=georgiaRed,bg=white}
\setbeamercolor{palette tertiary}{fg=georgiaRed,bg=white}
\setbeamercolor{palette quaternary}{bg=mediumGray,fg=black}
\setbeamercolor{block title}{fg=white,bg=georgiaRed}
\setbeamercolor{block body}{fg=black,bg=black!10}
\setbeamercolor{titlelike}{bg=georgiaRed,fg=white} % parent=palette quaternary}

% Define the variable to determine whether or not the clicker quizzes
% are visible in the resulting output.
\newtoggle{clicker}
\toggletrue{clicker}
%\togglefalse{clicker}


% To display a lecture uncomment out the "includeonly" line below to
% match the name of the file. You do not have to do anything with the
% lecture line below and can leave it commented out. It is in place
% because at one time we had multiple lectures within a file, but that
% has been changed.



\mode<presentation>{
  \setbeamercovered{invisible}
  \setbeameroption{hide notes}
}

\mode<handout>{
 
  \usepackage{pgfpages}
  %\pgfpagesuselayout{4 on 1}[letterpaper, border shrink=5mm]
  \pgfpagesuselayout{resize to}[letterpaper,border shrink=5mm]
  \setbeameroption{show notes}


  %\pgfpagesphysicalpageoptions{logical pages=2,physical
  %height=\pgfpageoptionheight,physical width=\pgfpageoptionwidth}
  % Set up the pages for notes.
  % This idea and some code came from
  % http://www.guidodiepen.nl/2009/07/creating-latex-beamer-handouts-with-notes/



  \pgfpagesdeclarelayout{3 on 1 with notes} {
    \edef\pgfpageoptionheight{11in} %\the\paperheight}
    \edef\pgfpageoptionwidth{8.5in} %\the\paperwidth}
    \edef\pgfpageoptionborder{0pt}
  }

  {

	\AtBeginDocument{
      \newbox\notesbox
      \setbox\notesbox=\vbox{
        \hsize=\paperwidth
        \vskip-2.5cm\hskip-5cm\vbox{
          \textcolor{light-gray}{\hrule width 12.6cm\vskip0.5cm}
          \textcolor{light-gray}{\hrule width 12.6cm\vskip0.5cm}
          \textcolor{light-gray}{\hrule width 12.6cm\vskip0.5cm}
          \textcolor{light-gray}{\hrule width 12.6cm\vskip0.5cm}
          \textcolor{light-gray}{\hrule width 12.6cm\vskip0.5cm}
          \textcolor{light-gray}{\hrule width 12.6cm\vskip0.5cm}
          \textcolor{light-gray}{\hrule width 12.6cm\vskip0.5cm}
          \textcolor{light-gray}{\hrule width 12.6cm\vskip0.5cm}
          \textcolor{light-gray}{\hrule width 12.6cm\vskip0.5cm}
          \textcolor{light-gray}{\hrule width 12.6cm\vskip0.5cm}
          \textcolor{light-gray}{\hrule width 12.6cm\vskip0.5cm}
          \textcolor{light-gray}{\hrule width 12.6cm\vskip0.5cm}
          \textcolor{light-gray}{\hrule width 12.6cm\vskip0.5cm}
          \textcolor{light-gray}{\hrule width 12.6cm\vskip0.5cm}
          \textcolor{light-gray}{\hrule width 12.6cm\vskip0.5cm}
          \textcolor{light-gray}{\hrule width 12.6cm\vskip0.5cm}
          \textcolor{light-gray}{\hrule width 12.6cm\vskip0.5cm}
          \textcolor{light-gray}{\hrule width 12.6cm\vskip0.5cm}
          \textcolor{light-gray}{\hrule width 12.6cm\vskip0.5cm}

          \vspace*{-9.75cm}
          \textcolor{light-gray}{\rule[-1.0cm]{1pt}{9.25cm}\hskip0.5cm}
          \textcolor{light-gray}{\rule[-1.0cm]{1pt}{9.25cm}\hskip0.5cm}
          \textcolor{light-gray}{\rule[-1.0cm]{1pt}{9.25cm}\hskip0.5cm}
          \textcolor{light-gray}{\rule[-1.0cm]{1pt}{9.25cm}\hskip0.5cm}
          \textcolor{light-gray}{\rule[-1.0cm]{1pt}{9.25cm}\hskip0.5cm}
          \textcolor{light-gray}{\rule[-1.0cm]{1pt}{9.25cm}\hskip0.5cm}
          \textcolor{light-gray}{\rule[-1.0cm]{1pt}{9.25cm}\hskip0.5cm}
          \textcolor{light-gray}{\rule[-1.0cm]{1pt}{9.25cm}\hskip0.5cm}
          \textcolor{light-gray}{\rule[-1.0cm]{1pt}{9.25cm}\hskip0.5cm}
          \textcolor{light-gray}{\rule[-1.0cm]{1pt}{9.25cm}\hskip0.5cm}
          \textcolor{light-gray}{\rule[-1.0cm]{1pt}{9.25cm}\hskip0.5cm}
          \textcolor{light-gray}{\rule[-1.0cm]{1pt}{9.25cm}\hskip0.5cm}
          \textcolor{light-gray}{\rule[-1.0cm]{1pt}{9.25cm}\hskip0.5cm}
          \textcolor{light-gray}{\rule[-1.0cm]{1pt}{9.25cm}\hskip0.5cm}
          \textcolor{light-gray}{\rule[-1.0cm]{1pt}{9.25cm}\hskip0.5cm}
          \textcolor{light-gray}{\rule[-1.0cm]{1pt}{9.25cm}\hskip0.5cm}
          \textcolor{light-gray}{\rule[-1.0cm]{1pt}{9.25cm}\hskip0.5cm}
          \textcolor{light-gray}{\rule[-1.0cm]{1pt}{9.25cm}\hskip0.5cm}
          \textcolor{light-gray}{\rule[-1.0cm]{1pt}{9.25cm}\hskip0.5cm}
          \textcolor{light-gray}{\rule[-1.0cm]{1pt}{9.25cm}\hskip0.5cm}

        }

      }

    \pgfpagesphysicalpageoptions
    {%
      logical pages=6,%
      physical height=\pgfpageoptionheight,%
      physical width=\pgfpageoptionwidth,%
      last logical shipout=3%
    }
    
    \pgfpageslogicalpageoptions{1}
    {%
      border shrink=\pgfpageoptionborder,%
      resized width=.5\pgfphysicalwidth,%
      resized height=.33\pgfphysicalheight,%
      center=\pgfpoint{.25\pgfphysicalwidth}{.82\pgfphysicalheight}%
    }%
    \pgfpageslogicalpageoptions{2}
    {%
      border shrink=\pgfpageoptionborder,%
      resized width=.5\pgfphysicalwidth,%
      resized height=.33\pgfphysicalheight,%
      center=\pgfpoint{.25\pgfphysicalwidth}{.47\pgfphysicalheight}%
    }%
    \pgfpageslogicalpageoptions{3}
    {%
      border shrink=\pgfpageoptionborder,%
      resized width=.5\pgfphysicalwidth,%
      resized height=.33\pgfphysicalheight,%
      center=\pgfpoint{.25\pgfphysicalwidth}{.17\pgfphysicalheight}%
    }%	
	\pgfpageslogicalpageoptions{4}
    {%
      border shrink=\pgfpageoptionborder,%
      resized width=.5\pgfphysicalwidth,%
      resized height=.33\pgfphysicalheight,%
      center=\pgfpoint{.85\pgfphysicalwidth}{.82\pgfphysicalheight},%
      copy from=4
    }%
    \pgfpageslogicalpageoptions{5}
    {%
      border shrink=\pgfpageoptionborder,%
      resized width=.5\pgfphysicalwidth,%
      resized height=.33\pgfphysicalheight,%
      center=\pgfpoint{.85\pgfphysicalwidth}{.47\pgfphysicalheight},%
      copy from=5
    }%
    \pgfpageslogicalpageoptions{6}
    {%
      border shrink=\pgfpageoptionborder,%
      resized width=.5\pgfphysicalwidth,%
      resized height=.33\pgfphysicalheight,%
      center=\pgfpoint{.85\pgfphysicalwidth}{.17\pgfphysicalheight},%
      copy from=6
    }%
    
      \pgfpagesshipoutlogicalpage{4}\copy\notesbox
      \pgfpagesshipoutlogicalpage{5}\copy\notesbox
      \pgfpagesshipoutlogicalpage{6}\copy\notesbox
    }
  }

  \pgfpagesuselayout{3 on 1 with notes}

}

\setbeamercolor{upper separation line head}{bg=red}
\setbeamercolor{headline}{bg=red}
\setbeamertemplate{headline}
{%
\begin{beamercolorbox}{section in head/foot}
\insertsectionnavigationhorizontal{.75\textwidth}{}{}
\hfill \insertpagenumber /\insertdocumentendpage
\end{beamercolorbox}%
}
\setbeamercolor{section number projected}{bg=red,fg=black}
\setbeamercolor{subsection number projected}{bg=red,fg=black}
%\setbeamercolor{frametitle}{bg=lightgray,fg=black}

\setbeamertemplate{itemize item}{\color{georgiaRed}$\blacklozenge$}
\setbeamertemplate{itemize subitem}{\color{georgiaRed}$\blacktriangleright$}

\newcommand{\dotfield}[2]{%
  \begin{tikzpicture}[y=0.25cm, x=0.25cm,font=\sffamily]
    \foreach \y in {0,...,#2} {
      \foreach \x in {0,...,#1} {
        \draw[fill=georgiaRed,opacity=0.1] (\x,\y)  circle [radius=0.03em];
      }
    }
  \end{tikzpicture}
}

\newcommand{\twoByTwo}[4]{%
  \left[
    \begin{array}{rr}
      #1 & #2 \\
      #3 & #4 \\
    \end{array}
  \right]
}

\newcommand{\threeByThree}[9]{%
  \left[
    \begin{array}{rrr}
      #1 & #2 & #3 \\
      #4 & #5 & #6 \\
      #7 & #8 & #9
    \end{array}
  \right]
}

\newcommand{\columnVector}[1]{%
  \left[
    \begin{array}{r}
    #1                           
    \end{array}
  \right]
}


\newcommand{\rowFour}[4]{
  #1 & #2 & #3 & #4
}

\newcommand{\threeRows}[3]{
  \left[
    \begin{array}{rrrr}
      #1 \\
      #2 \\
      #3 
    \end{array}
  \right]
}



\begin{document}



\author{\textsc{T. Alli$^{a}$, K. Black$^{a}$}}
\institute{$^a$Department of Mathematics, University of Georgia, GA}
\subject{Linear Algebra}
\keywords{Linear Transformation, Vectors, Matrices, Linear Algebra}

%\lecture{Partial Fractions}{partial-fractions}
%\section{Rational Functions}

\title{Section 2.5: Linear Indepenence}
\subtitle{Expressing Vectors As Linear Combinations}


\date{} % {\today}

\begin{frame}
  \titlepage
\end{frame}

\begin{frame}{Outline}
  \tableofcontents
\end{frame}


\section{Goals}

\begin{frame}{Goals}

  \begin{itemize}
  \item Determine whether or not a set of vectors is linearly
    independent or not.
  \item Given a set of vectors determine which ones can be expressed
    as a linear combination of the other vectors.
  \item Relate the reduced row echelon form of a matrix with the
    linear independence of its columns.
  \end{itemize}

\end{frame}

\section{Motivation For Linear Independence/Dependence}

\begin{frame}{Motivation For Linear Independence}

  Suppose we have three vectors:
  \begin{eqnarray*}
    \left\{
    \columnVector{1 \\ 0 \\ 0},
    \columnVector{0 \\ 1 \\ 0},
    \columnVector{0 \\ 0 \\ 1}
    \right\}
  \end{eqnarray*}

  \uncover<2->{
    Given another vector, I can express the new vector as a linear
    combination of these vectors:
    \begin{eqnarray*}
      \columnVector{ -3 \\ 2 \\ 4}
      & = &
      \uncover<3->{
         -3 \columnVector{1 \\ 0 \\ 0} +
          2 \columnVector{0 \\ 1 \\ 0} +
          4 \columnVector{0 \\ 0 \\ 1}
      }
    \end{eqnarray*}
  }

\end{frame}

\begin{frame}{Is It Always So Easy?}

  Suppose we have three vectors:
  \begin{eqnarray*}
    \left\{
    \columnVector{1 \\ 0 \\ 0},
    \columnVector{1 \\ 1 \\ 0},
    \columnVector{1 \\ 1 \\ 1}
    \right\}
  \end{eqnarray*}

  \only<2-3>{
    Given another vector, I can express the new vector as a linear
    combination of these vectors:
    \begin{eqnarray*}
      \columnVector{ -3 \\ 2 \\ 4}
      & = &
      \uncover<3>{
          a \columnVector{1 \\ 0 \\ 0} +
          b \columnVector{1 \\ 1 \\ 0} +
          c \columnVector{1 \\ 1 \\ 1}
      }
    \end{eqnarray*}
  }

  \only<4>
  {
    \begin{eqnarray*}
      \left[
      \begin{array}{rrr|r}
        1 & 1 & 1 & -3 \\
        0 & 1 & 1 &  2 \\
        0 & 0 & 1 &  4
      \end{array}
      \right]
    \end{eqnarray*}
  }

\end{frame}

\begin{frame}{Can I Do This For Any Vector?}

  Given a set of vectors can I determine whether or not I can write
  \textbf{any} vector as a linear combination of the vectors in the
  set?

  \vfill
  
  \uncover<2->{Excellent question, but this is not what we will do
    today! We will save this for later. Instead we ask a related
    question.}

  \vfill
  
\end{frame}

\begin{frame}{Can I Find Any Vector In A Set and Write As A Linear
    Combination Of The Others?}

  Suppose we have three vectors:
  \begin{eqnarray*}
    \left\{
    \columnVector{1 \\ 0 \\ 0},
    \columnVector{1 \\ 1 \\ 0},
    \columnVector{1 \\ 2 \\ 0}
    \right\}
  \end{eqnarray*}

  One of these vectors can be written in terms of the others:
  \begin{eqnarray*}
    \columnVector{ 1 \\ 2 \\ 0}
    & = &
          -1 \columnVector{1 \\ 0 \\ 0} +
    2 \columnVector{1 \\ 1 \\ 0} 
  \end{eqnarray*}


\end{frame}

\begin{frame}{Can I Find Any Vector In A Set and Write As A Linear
    Combination Of The Others?}

  Suppose we have three vectors:
  \begin{eqnarray*}
    \left\{
    \columnVector{1 \\ 0 \\ 0},
    \columnVector{1 \\ 1 \\ 0},
    \columnVector{1 \\ 2 \\ 3},
    \columnVector{1 \\ 2 \\ 1}
    \right\}
  \end{eqnarray*}

  Which one can be written in terms of some of the others? How many?

\end{frame}


\section{Linear Independence}

\begin{frame}{Can I Find Any Vector In A Set and Write As A Linear
    Combination Of The Others?}

  Suppose we have three vectors:
  \begin{eqnarray*}
    \left\{
    \columnVector{1 \\ 0 \\ 0},
    \columnVector{1 \\ 1 \\ 0},
    \columnVector{1 \\ 2 \\ 3},
    \columnVector{1 \\ 2 \\ 1}
    \right\}
  \end{eqnarray*}

  Look at a general linear combination:
  \begin{eqnarray*}
    x_1 \columnVector{1 \\ 0 \\ 0} +
    x_2 \columnVector{1 \\ 1 \\ 0} +
    x_3 \columnVector{1 \\ 2 \\ 3} +
    x_4 \columnVector{1 \\ 2 \\ 1}
    & = &
    \vec{0}.
  \end{eqnarray*}
  If one of these vectors can be written in terms of the others then
  the coefficient, $x_j$, does not \textbf{have} to be zero. If the
  \textbf{only} way this can work out is that $x_1=0$, $x_2=0$,
  $x_3=0$, and $x_4=0$ then there is no way to express one of the
  vectors as a linear combination of the others.
  

\end{frame}

\begin{frame}{Example}

  Can I express one of these vectors as a linear combination of the others?
    \begin{eqnarray*}
      \left\{
      \columnVector{ 1 \\   2 \\   7 \\   0},
      \columnVector{3  \\  7  \\  25 \\  -2},
      \columnVector{0  \\  2  \\  8  \\ -4},
      \columnVector{2  \\  4  \\ 17  \\ -3},
      \columnVector{1  \\  3  \\ 12  \\ -6},
      \columnVector{2  \\  8  \\ 28  \\ -4}
    \right\}
  \end{eqnarray*}

  \uncover<2->{
    \begin{eqnarray*}
      x_1 \columnVector{ 1 \\   2 \\   7 \\   0} + 
      x_2 \columnVector{3  \\  7  \\  25 \\  -2} +
      x_3 \columnVector{0  \\  2  \\  8  \\ -4} +
      x_4 \columnVector{2  \\  4  \\ 17  \\ -3} +
      x_5 \columnVector{1  \\  3  \\ 12  \\ -6} +
      x_6 \columnVector{2  \\  8  \\ 28  \\ -4}
      & = &
      \vec{0}.
    \end{eqnarray*}
  }
  
\end{frame}


\begin{frame}{Augmented Matrix}

  The augmented matrix for the system of linear equations,
  \begin{eqnarray*}
    x_1 \columnVector{ 1 \\   2 \\   7 \\   0} + 
    x_2 \columnVector{3  \\  7  \\  25 \\  -2} +
    x_3 \columnVector{0  \\  2  \\  8  \\ -4} +
    x_4 \columnVector{2  \\  4  \\ 17  \\ -3} +
    x_5 \columnVector{1  \\  3  \\ 12  \\ -6} +
    x_6 \columnVector{2  \\  8  \\ 28  \\ -4}
    & = &
    \vec{0},
  \end{eqnarray*}

  \only<2>{
  \begin{eqnarray*}
    \left[
    \begin{array}{rrrrrr}
      1 &  3 &  0 &  2 &  1 & 2 \\
      2 &  7 &  2 &  4 &  3 & 8 \\
      7 & 25 &  8 & 17 & 12 &28 \\
      0 & -2 & -4 & -3 & -6 &-4
    \end{array}
    \right]
  \end{eqnarray*}
  }
  \only<3>{
  \begin{eqnarray*}
    \left[
    \begin{array}{rrrrrr}
      1 & 3 & 0 & 2 & 1 & 2 \\
      0 & 1 & 2 & 0 & 1 & 4 \\
      0 & 0 & 0 & 3 & 1 &-2 \\
      0 & 0 & 0 & 0 &-3 & 2
    \end{array}
    \right]
  \end{eqnarray*}
  }

\end{frame}

\begin{frame}{Solving}
    \begin{eqnarray*}
    \left[
    \begin{array}{rrrrrr}
      1 & 3 & 0 & 2 & 1 & 2 \\
      0 & 1 & 2 & 0 & 1 & 4 \\
      0 & 0 & 0 & 3 & 1 &-2 \\
      0 & 0 & 0 & 0 &-3 & 2
    \end{array}
    \right]
  \end{eqnarray*}

  \dotfield{60}{18}
  
\end{frame}

\begin{frame}{Results}

  \begin{columns}
    \column{0.3\textwidth}

    \begin{eqnarray*}
      x_1 & = & 6 x_3 +  \frac{94}{9} x_6 \\
      x_2 & = & -2 x_3 + \frac{-14}{3} x_6 \\
      x_3 & = & x_3 \\
      x_4 & = & \frac{4}{9} x_6 \\
      x_5 & = & \frac{2}{3} x_6 \\
      x_6 & = & x_6
    \end{eqnarray*}

    \column{0.7\textwidth}

    \begin{eqnarray*}
      x_1 \columnVector{ 1 \\   2 \\   7 \\   0} + 
      x_2 \columnVector{3  \\  7  \\  25 \\  -2} +
      x_3 \columnVector{0  \\  2  \\  8  \\ -4} +  & & \\
      x_4 \columnVector{2  \\  4  \\ 17  \\ -3} + 
      x_5 \columnVector{1  \\  3  \\ 12  \\ -6} +
      x_6 \columnVector{2  \\  8  \\ 28  \\ -4}
      & = &
    \vec{0},
  \end{eqnarray*}

  \end{columns}
  
\end{frame}

\begin{frame}{Definition}

  \begin{block}{Linear Independence}
    A set of vectors,
    $\left\{ \vec{v}_1, \vec{v}_2, \ldots , \vec{v}_k \right\}$, is
    linearly independent if and only if the \textbf{only} solution to
    the vector equation
    \begin{eqnarray*}
      x_1 \vec{v}_1 + x_2 \vec{v}_2 +  \cdots + x_k \vec{v}_k
      & = & \vec{0}
    \end{eqnarray*}
    is $x_1=0$, $x_2=0$, $\ldots$, $x_k=0$.
  \end{block}
  
  
\end{frame}

\section{Geometric View}

\begin{frame}{Geometric View}
  \dotfield{60}{24}
\end{frame}

\begin{frame}{Geometric View}
  \dotfield{60}{24}
\end{frame}

\begin{frame}{Geometric View}
  \dotfield{60}{24}
\end{frame}


\begin{frame}{}
  \dotfield{60}{24}
\end{frame}


\end{document}
