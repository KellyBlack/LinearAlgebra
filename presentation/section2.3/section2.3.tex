\documentclass[svgnames,table,,aspectratio=169]{beamer}
%\documentclass[svgnames,table,handout,aspectratio=129]{beamer}
\usepackage{hhline}
\usepackage{etoolbox}
\usepackage{tikz}
\usepackage{mathtools}
\usepackage{amssymb}
%\usepackage{/usr/lib64/R/share/texmf/Sweave}
\usepackage{polynom}
\usepackage{qrcode}


%\input{latexdefinitions}
\definecolor{georgiaRed}{RGB}{100,0,00}
\definecolor{mediumGray}{gray}{0.6}



\usetheme{Frankfurt}%
%\usetheme{Warsaw}%
%\useoutertheme{smoothbars}


%\usecolortheme{seagull}
\usecolortheme{beaver}
\logo{\includegraphics[height=.125in]{ugaLogo}}

% Note that the colour definitions are given in the latexDefinitions
% file.
\setbeamercolor{palette primary}{fg=georgiaRed,bg=white}
\setbeamercolor{palette secondary}{fg=georgiaRed,bg=white}
\setbeamercolor{palette tertiary}{fg=georgiaRed,bg=white}
\setbeamercolor{palette quaternary}{bg=mediumGray,fg=black}
\setbeamercolor{block title}{fg=black,bg=black!15}
\setbeamercolor{block body}{fg=black,bg=black!10}
\setbeamercolor{titlelike}{bg=georgiaRed,fg=white} % parent=palette quaternary}

% Define the variable to determine whether or not the clicker quizzes
% are visible in the resulting output.
\newtoggle{clicker}
\toggletrue{clicker}
%\togglefalse{clicker}


% To display a lecture uncomment out the "includeonly" line below to
% match the name of the file. You do not have to do anything with the
% lecture line below and can leave it commented out. It is in place
% because at one time we had multiple lectures within a file, but that
% has been changed.



\mode<presentation>{
  \setbeamercovered{invisible}
  \setbeameroption{hide notes}
}

\mode<handout>{
 
  \usepackage{pgfpages}
  %\pgfpagesuselayout{4 on 1}[letterpaper, border shrink=5mm]
  \pgfpagesuselayout{resize to}[letterpaper,border shrink=5mm]
  \setbeameroption{show notes}


  %\pgfpagesphysicalpageoptions{logical pages=2,physical
  %height=\pgfpageoptionheight,physical width=\pgfpageoptionwidth}
  % Set up the pages for notes.
  % This idea and some code came from
  % http://www.guidodiepen.nl/2009/07/creating-latex-beamer-handouts-with-notes/



  \pgfpagesdeclarelayout{3 on 1 with notes} {
    \edef\pgfpageoptionheight{11in} %\the\paperheight}
    \edef\pgfpageoptionwidth{8.5in} %\the\paperwidth}
    \edef\pgfpageoptionborder{0pt}
  }

  {

	\AtBeginDocument{
      \newbox\notesbox
      \setbox\notesbox=\vbox{
        \hsize=\paperwidth
        \vskip-2.5cm\hskip-5cm\vbox{
          \textcolor{light-gray}{\hrule width 12.6cm\vskip0.5cm}
          \textcolor{light-gray}{\hrule width 12.6cm\vskip0.5cm}
          \textcolor{light-gray}{\hrule width 12.6cm\vskip0.5cm}
          \textcolor{light-gray}{\hrule width 12.6cm\vskip0.5cm}
          \textcolor{light-gray}{\hrule width 12.6cm\vskip0.5cm}
          \textcolor{light-gray}{\hrule width 12.6cm\vskip0.5cm}
          \textcolor{light-gray}{\hrule width 12.6cm\vskip0.5cm}
          \textcolor{light-gray}{\hrule width 12.6cm\vskip0.5cm}
          \textcolor{light-gray}{\hrule width 12.6cm\vskip0.5cm}
          \textcolor{light-gray}{\hrule width 12.6cm\vskip0.5cm}
          \textcolor{light-gray}{\hrule width 12.6cm\vskip0.5cm}
          \textcolor{light-gray}{\hrule width 12.6cm\vskip0.5cm}
          \textcolor{light-gray}{\hrule width 12.6cm\vskip0.5cm}
          \textcolor{light-gray}{\hrule width 12.6cm\vskip0.5cm}
          \textcolor{light-gray}{\hrule width 12.6cm\vskip0.5cm}
          \textcolor{light-gray}{\hrule width 12.6cm\vskip0.5cm}
          \textcolor{light-gray}{\hrule width 12.6cm\vskip0.5cm}
          \textcolor{light-gray}{\hrule width 12.6cm\vskip0.5cm}
          \textcolor{light-gray}{\hrule width 12.6cm\vskip0.5cm}

          \vspace*{-9.75cm}
          \textcolor{light-gray}{\rule[-1.0cm]{1pt}{9.25cm}\hskip0.5cm}
          \textcolor{light-gray}{\rule[-1.0cm]{1pt}{9.25cm}\hskip0.5cm}
          \textcolor{light-gray}{\rule[-1.0cm]{1pt}{9.25cm}\hskip0.5cm}
          \textcolor{light-gray}{\rule[-1.0cm]{1pt}{9.25cm}\hskip0.5cm}
          \textcolor{light-gray}{\rule[-1.0cm]{1pt}{9.25cm}\hskip0.5cm}
          \textcolor{light-gray}{\rule[-1.0cm]{1pt}{9.25cm}\hskip0.5cm}
          \textcolor{light-gray}{\rule[-1.0cm]{1pt}{9.25cm}\hskip0.5cm}
          \textcolor{light-gray}{\rule[-1.0cm]{1pt}{9.25cm}\hskip0.5cm}
          \textcolor{light-gray}{\rule[-1.0cm]{1pt}{9.25cm}\hskip0.5cm}
          \textcolor{light-gray}{\rule[-1.0cm]{1pt}{9.25cm}\hskip0.5cm}
          \textcolor{light-gray}{\rule[-1.0cm]{1pt}{9.25cm}\hskip0.5cm}
          \textcolor{light-gray}{\rule[-1.0cm]{1pt}{9.25cm}\hskip0.5cm}
          \textcolor{light-gray}{\rule[-1.0cm]{1pt}{9.25cm}\hskip0.5cm}
          \textcolor{light-gray}{\rule[-1.0cm]{1pt}{9.25cm}\hskip0.5cm}
          \textcolor{light-gray}{\rule[-1.0cm]{1pt}{9.25cm}\hskip0.5cm}
          \textcolor{light-gray}{\rule[-1.0cm]{1pt}{9.25cm}\hskip0.5cm}
          \textcolor{light-gray}{\rule[-1.0cm]{1pt}{9.25cm}\hskip0.5cm}
          \textcolor{light-gray}{\rule[-1.0cm]{1pt}{9.25cm}\hskip0.5cm}
          \textcolor{light-gray}{\rule[-1.0cm]{1pt}{9.25cm}\hskip0.5cm}
          \textcolor{light-gray}{\rule[-1.0cm]{1pt}{9.25cm}\hskip0.5cm}

        }

      }

    \pgfpagesphysicalpageoptions
    {%
      logical pages=6,%
      physical height=\pgfpageoptionheight,%
      physical width=\pgfpageoptionwidth,%
      last logical shipout=3%
    }
    
    \pgfpageslogicalpageoptions{1}
    {%
      border shrink=\pgfpageoptionborder,%
      resized width=.5\pgfphysicalwidth,%
      resized height=.33\pgfphysicalheight,%
      center=\pgfpoint{.25\pgfphysicalwidth}{.82\pgfphysicalheight}%
    }%
    \pgfpageslogicalpageoptions{2}
    {%
      border shrink=\pgfpageoptionborder,%
      resized width=.5\pgfphysicalwidth,%
      resized height=.33\pgfphysicalheight,%
      center=\pgfpoint{.25\pgfphysicalwidth}{.47\pgfphysicalheight}%
    }%
    \pgfpageslogicalpageoptions{3}
    {%
      border shrink=\pgfpageoptionborder,%
      resized width=.5\pgfphysicalwidth,%
      resized height=.33\pgfphysicalheight,%
      center=\pgfpoint{.25\pgfphysicalwidth}{.17\pgfphysicalheight}%
    }%	
	\pgfpageslogicalpageoptions{4}
    {%
      border shrink=\pgfpageoptionborder,%
      resized width=.5\pgfphysicalwidth,%
      resized height=.33\pgfphysicalheight,%
      center=\pgfpoint{.85\pgfphysicalwidth}{.82\pgfphysicalheight},%
      copy from=4
    }%
    \pgfpageslogicalpageoptions{5}
    {%
      border shrink=\pgfpageoptionborder,%
      resized width=.5\pgfphysicalwidth,%
      resized height=.33\pgfphysicalheight,%
      center=\pgfpoint{.85\pgfphysicalwidth}{.47\pgfphysicalheight},%
      copy from=5
    }%
    \pgfpageslogicalpageoptions{6}
    {%
      border shrink=\pgfpageoptionborder,%
      resized width=.5\pgfphysicalwidth,%
      resized height=.33\pgfphysicalheight,%
      center=\pgfpoint{.85\pgfphysicalwidth}{.17\pgfphysicalheight},%
      copy from=6
    }%
    
      \pgfpagesshipoutlogicalpage{4}\copy\notesbox
      \pgfpagesshipoutlogicalpage{5}\copy\notesbox
      \pgfpagesshipoutlogicalpage{6}\copy\notesbox
    }
  }

  \pgfpagesuselayout{3 on 1 with notes}

}

\setbeamercolor{upper separation line head}{bg=red}
\setbeamercolor{headline}{bg=red}
\setbeamertemplate{headline}
{%
\begin{beamercolorbox}{section in head/foot}
\insertsectionnavigationhorizontal{.75\textwidth}{}{}
\hfill \insertpagenumber /\insertdocumentendpage
\end{beamercolorbox}%
}
\setbeamercolor{section number projected}{bg=red,fg=black}
\setbeamercolor{subsection number projected}{bg=red,fg=black}
%\setbeamercolor{frametitle}{bg=lightgray,fg=black}

\setbeamertemplate{itemize item}{\color{georgiaRed}$\blacklozenge$}
\setbeamertemplate{itemize subitem}{\color{georgiaRed}$\blacktriangleright$}

\newcommand{\dotfield}[2]{%
  \begin{tikzpicture}[y=0.25cm, x=0.25cm,font=\sffamily]
    \foreach \y in {0,...,#2} {
      \foreach \x in {0,...,#1} {
        \draw[fill=georgiaRed,opacity=0.1] (\x,\y)  circle [radius=0.03em];
      }
    }
  \end{tikzpicture}
}

\newcommand{\twoByTwo}[4]{%
  \left[
    \begin{array}{rr}
      #1 & #2 \\
      #3 & #4 \\
    \end{array}
  \right]
}

\newcommand{\threeByThree}[9]{%
  \left[
    \begin{array}{rrr}
      #1 & #2 & #3 \\
      #4 & #5 & #6 \\
      #7 & #8 & #9
    \end{array}
  \right]
}

\newcommand{\columnVector}[1]{%
  \left[
    \begin{array}{r}
    #1                           
    \end{array}
  \right]
}


\begin{document}



\author{\textsc{T. Alli$^{a}$, K. Black$^{a}$}}
\institute{$^a$Department of Mathematics, University of Georgia, GA}
\subject{Linear Algebra}
\keywords{Linear Transformation, Vectors, Matrices, Linear Algebra}

%\lecture{Partial Fractions}{partial-fractions}
%\section{Rational Functions}

\title{Section 2.3: Matrix Equations}
\subtitle{How To Determine Whether Or Not A Solution To A Linear
  System Exists}


\date{} % {\today}

\begin{frame}
  \titlepage
\end{frame}

\begin{frame}{Outline}
  \tableofcontents
\end{frame}


\section{Goals}

\begin{frame}{Goals}

  \begin{itemize}
  \item Express a given system of linear equations four different
    ways, including using a matrix.
  \item Determine if a system of linear equations is consistent given the
    matrix associated with the system of equations.
  \item Perform basic matrix/vector operations.
  \item Determine conditions necessary for a solution to always exist
    given the matrix associated with the system of equations.
  \end{itemize}

\end{frame}

\section{Matrices}

\begin{frame}{What is a matrix?}

  A matrix is a set of numbers in an array with $m$ rows and $n$
  columns,
  \begin{eqnarray*}
    A & = &
            \left[
            \begin{array}{rrrrrr}
              a_{1,1} & a_{1,2} & a_{1,3} & \cdots & a_{1,n-1} & a_{1,n} \\
              a_{2,1} & a_{2,2} & a_{2,3} & \cdots & a_{2,n-1} & a_{2,n} \\
              a_{3,1} & a_{3,2} & a_{3,3} & \cdots & a_{3,n-1} & a_{3,n} \\
              \vdots  & \vdots & \vdots &        & \vdots   & \vdots \\
              a_{m,1} & a_{m,2} & a_{m,3} & \cdots & a_{m,n-1} & a_{m,n} 
            \end{array}
            \right].
  \end{eqnarray*}

\end{frame}

\begin{frame}{Example}

  For example, a matrix with 4 rows and 3 columns is shown below:
  \only<1>{
    \begin{eqnarray*}
      A & = &
              \left[
              \begin{array}{rrr}
                -4 &  2 &  1 \\
                 8 & -2 &  0 \\
                 3 &  1 &  7 \\
                -9 &  6 &  4
              \end{array}
                          \right].
    \end{eqnarray*}
    }
  \only<2->{
    \begin{eqnarray*}
      A & = &
              \left[
              \begin{array}{r|r|r}
                -4 &  2 &  1 \\
                 8 & -2 &  0 \\
                 3 &  1 &  7 \\
                -9 &  6 &  4
              \end{array}
                          \right].
    \end{eqnarray*}
    }
  \uncover<3>{
    \begin{eqnarray*}
      A & = &
              \left[
              \begin{array}{r|r|r}
                \vec{v}_1 & \vec{v}_2 & \vec{v}_3
              \end{array}
                          \right].
    \end{eqnarray*}
    }

  Nomenclature: In this case we say that $A$ is a $4\times 3$
  matrix. We may also say that $A$ is in ${\mathbb R}^{4\times 3}$.
  
  
\end{frame}

\begin{frame}{What is a matrix?}

  A matrix is a set of numbers in an array with $m$ rows and $n$
  columns,
    \begin{eqnarray*}
      A & = &
              \left[
              \begin{array}{rrrrrr}
                a_{1,1} & a_{1,2} & a_{1,3} & \cdots & a_{1,n-1} & a_{1,n} \\
                a_{2,1} & a_{2,2} & a_{2,3} & \cdots & a_{2,n-1} & a_{2,n} \\
                a_{3,1} & a_{3,2} & a_{3,3} & \cdots & a_{3,n-1} & a_{3,n} \\
                \vdots  & \vdots & \vdots &        & \vdots   & \vdots \\
                a_{m,1} & a_{m,2} & a_{m,3} & \cdots & a_{m,n-1} & a_{m,n} 
              \end{array}
              \right].
    \end{eqnarray*}
  \uncover<2>{
    \begin{eqnarray*}
      A & = &
              \left[
              \begin{array}{rrrrrr}
                \vec{v}_1 & \vec{v}_2 & \vec{v}_3 & \cdots &
                \vec{v}_{n-1} & \vec{v}_n
              \end{array}
              \right].
    \end{eqnarray*}
    Each vector is a column vector with $m$ rows.
  }


\end{frame}


\begin{frame}{Why Go To All This Trouble?}

  We define matrix/vector multiplication to be consistent with the
  idea of vector equations we saw in the previous section.

  Recall:
  \begin{eqnarray*}
    2 x_1 - x_2 + 4 x_3 & = & 0, \\
    4 x_1 + x_2 - 5 x_3 & = & 20.
  \end{eqnarray*}
  \uncover<2->{
    This can be written as
    \begin{eqnarray*}
      x_1 \columnVector{ 2 \\ 4} +
      x_2 \columnVector{-1 \\ 1} +
      x_3 \columnVector{4 \\ -5}
      & = &
      \columnVector{0 \\ 20}.
    \end{eqnarray*}
  }
  
\end{frame}

\begin{frame}{Matrix/Vector Multiplication}

  We define matrix/vector multiplication so that
  \begin{eqnarray*}
  \left[
    \begin{array}{rrr}
      2 & -1 &  4 \\
      4 &  1 & -5 \\
    \end{array}
    \right]
    \columnVector{x_1 \\ x_2 \\ x_3}
    & = &
      x_1 \columnVector{ 2 \\ 4} +
      x_2 \columnVector{-1 \\ 1} +
      x_3 \columnVector{4 \\ -5}              
  \end{eqnarray*}
  
\end{frame}

\begin{frame}{Equivalent Ways To Express A System Of Linear Equations}

    \begin{eqnarray*}
    2 x_1 - x_2 + 4 x_3 & = & 0, \\
    4 x_1 + x_2 - 5 x_3 & = & 20.
  \end{eqnarray*}

  This can be written as
  \begin{eqnarray*}
    x_1 \columnVector{ 2 \\ 4} +
    x_2 \columnVector{-1 \\ 1} +
    x_3 \columnVector{4 \\ -5}
    & = &
          \columnVector{0 \\ 20}.
  \end{eqnarray*}

  It can also be written as 
  \begin{eqnarray*}
    \left[
    \begin{array}{rrr}
      2 & -1 &  4 \\
      4 &  1 & -5 \\
    \end{array}
    \right]
    \columnVector{x_1 \\ x_2 \\ x_3}
    & = &
    \columnVector{0 \\ 20}.
  \end{eqnarray*}


\end{frame}

\begin{frame}{That Is Not What My High School Teacher Told Me!!!!}
  \vspace*{-2em}
  \begin{eqnarray*}
    \left[
    \begin{array}{rrr}
      2 & -1 &  4 \\
      4 &  1 & -5 \\
    \end{array}
    \right]
    \columnVector{x_1 \\ x_2 \\ x_3}
    & = &
    \columnVector{0 \\ 20}.
  \end{eqnarray*}

  \dotfield{60}{24}
  
\end{frame}

\begin{frame}{Details}

  By the way, we just defined a new operation! Surely, there must be
  some rules and restrictions? (Yes, of course there are.)

  We have
  \begin{eqnarray*}
    A\vec{x} & = & \vec{b}.
  \end{eqnarray*}
  In order to make sense the rows and columns have to work out. If $A$
  has $m$ rows and $n$ columns, then $\vec{x}$ must have $n$ rows, and
  $\vec{b}$ must have $m$ rows.
  
\end{frame}

\section{Matrix/Vector Operations}

\begin{frame}{Matrix/Vector Operations}

  The properties of vector operations translate nicely to matrix/vector operations:
  \begin{eqnarray*}
    A\left(r\vec{u}\right) & = & r A \vec{u}, \\
    A\left(\vec{u}+\vec{v}\right) & = & A\vec{u} + A\vec{v}.
  \end{eqnarray*}
  
\end{frame}

\section{Solutions To Linear Equations}

\begin{frame}{Solutions To Linear Equations}

  Given a matrix, $A$, and a vector $\vec{b}$ does a solution(s),
  $\vec{x}$, exist to the linear system
  \begin{eqnarray*}
    A\vec{x} & = & \vec{b}?
  \end{eqnarray*}
  \only<1>{
    \begin{itemize}
    \item Does $\vec{x}$ exist?
    \item What are all possible solutions?
    \item What are the restrictions on $\vec{b}$?
    \end{itemize}
  }

  \only<2>{
    \dotfield{60}{21}
  }
  
\end{frame}

\begin{frame}{Existence Of Solutions}

  \begin{eqnarray*}
    A & = & \left[ \vec{a}_1 ~ \vec{a}_2 ~ \vec{a}_3 ~ \cdots ~ \vec{a}_n \right]
  \end{eqnarray*}

  \dotfield{60}{24}
  
\end{frame}

\begin{frame}{Existence Of Solutions}

  For solutions to exist for the system of linear equations,
  \begin{eqnarray*}
    A \vec{x} & = & \vec{b},
  \end{eqnarray*}
  \begin{itemize}
  \item $x_1\vec{a}_1 ~ + ~ x_2 \vec{a}_2 ~ + ~ x_3 \vec{a}_3 ~ \cdots ~ + ~ x_n\vec{a}_n = \vec{b}$.
  \item $\vec{b}$ is a linear combination of the columns of $A$.
  \item $\vec{b}$ is in the span of the columns of $A$.
  \end{itemize}
  
\end{frame}

\begin{frame}{Existence Of Solutions}

  For solutions to exist for \textbf{ANY}  $\vec{b}$ the system of linear equations,
  \begin{eqnarray*}
    A \vec{x} & = & \vec{b},
  \end{eqnarray*}
  \begin{itemize}
  \item The span of the columns of $A$ is ${\mathbb R}^m$
  \item There is a pivot position in every row of $A$.
  \end{itemize}
  
\end{frame}


\begin{frame}{Blank Page}
  \dotfield{60}{24}
\end{frame}


\end{document}
