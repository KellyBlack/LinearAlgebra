\documentclass[svgnames,table,,aspectratio=169]{beamer}
%\documentclass[svgnames,table,handout,aspectratio=129]{beamer}
\usepackage{hhline}
\usepackage{etoolbox}
\usepackage{tikz}
\usepackage{mathtools}
\usepackage{amssymb}
%\usepackage{/usr/lib64/R/share/texmf/Sweave}
\usepackage{polynom}
\usepackage{qrcode}


%\input{latexdefinitions}
\definecolor{georgiaRed}{RGB}{100,0,00}
\definecolor{mediumGray}{gray}{0.6}



\usetheme{Frankfurt}%
%\usetheme{Warsaw}%
%\useoutertheme{smoothbars}


%\usecolortheme{seagull}
\usecolortheme{beaver}
\logo{\includegraphics[height=.125in]{ugaLogo}}

% Note that the colour definitions are given in the latexDefinitions
% file.
\setbeamercolor{palette primary}{fg=georgiaRed,bg=white}
\setbeamercolor{palette secondary}{fg=georgiaRed,bg=white}
\setbeamercolor{palette tertiary}{fg=georgiaRed,bg=white}
\setbeamercolor{palette quaternary}{bg=mediumGray,fg=black}
\setbeamercolor{block title}{fg=white,bg=georgiaRed}
\setbeamercolor{block body}{fg=black,bg=black!10}
\setbeamercolor{titlelike}{bg=georgiaRed,fg=white} % parent=palette quaternary}

% Define the variable to determine whether or not the clicker quizzes
% are visible in the resulting output.
\newtoggle{clicker}
\toggletrue{clicker}
%\togglefalse{clicker}


% To display a lecture uncomment out the "includeonly" line below to
% match the name of the file. You do not have to do anything with the
% lecture line below and can leave it commented out. It is in place
% because at one time we had multiple lectures within a file, but that
% has been changed.



\mode<presentation>{
  \setbeamercovered{invisible}
  \setbeameroption{hide notes}
}

\mode<handout>{
 
  \usepackage{pgfpages}
  %\pgfpagesuselayout{4 on 1}[letterpaper, border shrink=5mm]
  \pgfpagesuselayout{resize to}[letterpaper,border shrink=5mm]
  \setbeameroption{show notes}


  %\pgfpagesphysicalpageoptions{logical pages=2,physical
  %height=\pgfpageoptionheight,physical width=\pgfpageoptionwidth}
  % Set up the pages for notes.
  % This idea and some code came from
  % http://www.guidodiepen.nl/2009/07/creating-latex-beamer-handouts-with-notes/



  \pgfpagesdeclarelayout{3 on 1 with notes} {
    \edef\pgfpageoptionheight{11in} %\the\paperheight}
    \edef\pgfpageoptionwidth{8.5in} %\the\paperwidth}
    \edef\pgfpageoptionborder{0pt}
  }

  {

	\AtBeginDocument{
      \newbox\notesbox
      \setbox\notesbox=\vbox{
        \hsize=\paperwidth
        \vskip-2.5cm\hskip-5cm\vbox{
          \textcolor{light-gray}{\hrule width 12.6cm\vskip0.5cm}
          \textcolor{light-gray}{\hrule width 12.6cm\vskip0.5cm}
          \textcolor{light-gray}{\hrule width 12.6cm\vskip0.5cm}
          \textcolor{light-gray}{\hrule width 12.6cm\vskip0.5cm}
          \textcolor{light-gray}{\hrule width 12.6cm\vskip0.5cm}
          \textcolor{light-gray}{\hrule width 12.6cm\vskip0.5cm}
          \textcolor{light-gray}{\hrule width 12.6cm\vskip0.5cm}
          \textcolor{light-gray}{\hrule width 12.6cm\vskip0.5cm}
          \textcolor{light-gray}{\hrule width 12.6cm\vskip0.5cm}
          \textcolor{light-gray}{\hrule width 12.6cm\vskip0.5cm}
          \textcolor{light-gray}{\hrule width 12.6cm\vskip0.5cm}
          \textcolor{light-gray}{\hrule width 12.6cm\vskip0.5cm}
          \textcolor{light-gray}{\hrule width 12.6cm\vskip0.5cm}
          \textcolor{light-gray}{\hrule width 12.6cm\vskip0.5cm}
          \textcolor{light-gray}{\hrule width 12.6cm\vskip0.5cm}
          \textcolor{light-gray}{\hrule width 12.6cm\vskip0.5cm}
          \textcolor{light-gray}{\hrule width 12.6cm\vskip0.5cm}
          \textcolor{light-gray}{\hrule width 12.6cm\vskip0.5cm}
          \textcolor{light-gray}{\hrule width 12.6cm\vskip0.5cm}

          \vspace*{-9.75cm}
          \textcolor{light-gray}{\rule[-1.0cm]{1pt}{9.25cm}\hskip0.5cm}
          \textcolor{light-gray}{\rule[-1.0cm]{1pt}{9.25cm}\hskip0.5cm}
          \textcolor{light-gray}{\rule[-1.0cm]{1pt}{9.25cm}\hskip0.5cm}
          \textcolor{light-gray}{\rule[-1.0cm]{1pt}{9.25cm}\hskip0.5cm}
          \textcolor{light-gray}{\rule[-1.0cm]{1pt}{9.25cm}\hskip0.5cm}
          \textcolor{light-gray}{\rule[-1.0cm]{1pt}{9.25cm}\hskip0.5cm}
          \textcolor{light-gray}{\rule[-1.0cm]{1pt}{9.25cm}\hskip0.5cm}
          \textcolor{light-gray}{\rule[-1.0cm]{1pt}{9.25cm}\hskip0.5cm}
          \textcolor{light-gray}{\rule[-1.0cm]{1pt}{9.25cm}\hskip0.5cm}
          \textcolor{light-gray}{\rule[-1.0cm]{1pt}{9.25cm}\hskip0.5cm}
          \textcolor{light-gray}{\rule[-1.0cm]{1pt}{9.25cm}\hskip0.5cm}
          \textcolor{light-gray}{\rule[-1.0cm]{1pt}{9.25cm}\hskip0.5cm}
          \textcolor{light-gray}{\rule[-1.0cm]{1pt}{9.25cm}\hskip0.5cm}
          \textcolor{light-gray}{\rule[-1.0cm]{1pt}{9.25cm}\hskip0.5cm}
          \textcolor{light-gray}{\rule[-1.0cm]{1pt}{9.25cm}\hskip0.5cm}
          \textcolor{light-gray}{\rule[-1.0cm]{1pt}{9.25cm}\hskip0.5cm}
          \textcolor{light-gray}{\rule[-1.0cm]{1pt}{9.25cm}\hskip0.5cm}
          \textcolor{light-gray}{\rule[-1.0cm]{1pt}{9.25cm}\hskip0.5cm}
          \textcolor{light-gray}{\rule[-1.0cm]{1pt}{9.25cm}\hskip0.5cm}
          \textcolor{light-gray}{\rule[-1.0cm]{1pt}{9.25cm}\hskip0.5cm}

        }

      }

    \pgfpagesphysicalpageoptions
    {%
      logical pages=6,%
      physical height=\pgfpageoptionheight,%
      physical width=\pgfpageoptionwidth,%
      last logical shipout=3%
    }
    
    \pgfpageslogicalpageoptions{1}
    {%
      border shrink=\pgfpageoptionborder,%
      resized width=.5\pgfphysicalwidth,%
      resized height=.33\pgfphysicalheight,%
      center=\pgfpoint{.25\pgfphysicalwidth}{.82\pgfphysicalheight}%
    }%
    \pgfpageslogicalpageoptions{2}
    {%
      border shrink=\pgfpageoptionborder,%
      resized width=.5\pgfphysicalwidth,%
      resized height=.33\pgfphysicalheight,%
      center=\pgfpoint{.25\pgfphysicalwidth}{.47\pgfphysicalheight}%
    }%
    \pgfpageslogicalpageoptions{3}
    {%
      border shrink=\pgfpageoptionborder,%
      resized width=.5\pgfphysicalwidth,%
      resized height=.33\pgfphysicalheight,%
      center=\pgfpoint{.25\pgfphysicalwidth}{.17\pgfphysicalheight}%
    }%	
	\pgfpageslogicalpageoptions{4}
    {%
      border shrink=\pgfpageoptionborder,%
      resized width=.5\pgfphysicalwidth,%
      resized height=.33\pgfphysicalheight,%
      center=\pgfpoint{.85\pgfphysicalwidth}{.82\pgfphysicalheight},%
      copy from=4
    }%
    \pgfpageslogicalpageoptions{5}
    {%
      border shrink=\pgfpageoptionborder,%
      resized width=.5\pgfphysicalwidth,%
      resized height=.33\pgfphysicalheight,%
      center=\pgfpoint{.85\pgfphysicalwidth}{.47\pgfphysicalheight},%
      copy from=5
    }%
    \pgfpageslogicalpageoptions{6}
    {%
      border shrink=\pgfpageoptionborder,%
      resized width=.5\pgfphysicalwidth,%
      resized height=.33\pgfphysicalheight,%
      center=\pgfpoint{.85\pgfphysicalwidth}{.17\pgfphysicalheight},%
      copy from=6
    }%
    
      \pgfpagesshipoutlogicalpage{4}\copy\notesbox
      \pgfpagesshipoutlogicalpage{5}\copy\notesbox
      \pgfpagesshipoutlogicalpage{6}\copy\notesbox
    }
  }

  \pgfpagesuselayout{3 on 1 with notes}

}

\setbeamercolor{upper separation line head}{bg=red}
\setbeamercolor{headline}{bg=red}
\setbeamertemplate{headline}
{%
\begin{beamercolorbox}{section in head/foot}
\insertsectionnavigationhorizontal{.75\textwidth}{}{}
\hfill \insertpagenumber /\insertdocumentendpage
\end{beamercolorbox}%
}
\setbeamercolor{section number projected}{bg=red,fg=black}
\setbeamercolor{subsection number projected}{bg=red,fg=black}
%\setbeamercolor{frametitle}{bg=lightgray,fg=black}

\setbeamertemplate{itemize item}{\color{georgiaRed}$\blacklozenge$}
\setbeamertemplate{itemize subitem}{\color{georgiaRed}$\blacktriangleright$}

\newcommand{\dotfield}[2]{%
  \begin{tikzpicture}[y=0.25cm, x=0.25cm,font=\sffamily]
    \foreach \y in {0,...,#2} {
      \foreach \x in {0,...,#1} {
        \draw[fill=georgiaRed,opacity=0.1] (\x,\y)  circle [radius=0.03em];
      }
    }
  \end{tikzpicture}
}

\newcommand{\twoByTwo}[4]{%
  \left[
    \begin{array}{rr}
      #1 & #2 \\
      #3 & #4 \\
    \end{array}
  \right]
}

\newcommand{\threeByThree}[9]{%
  \left[
    \begin{array}{rrr}
      #1 & #2 & #3 \\
      #4 & #5 & #6 \\
      #7 & #8 & #9
    \end{array}
  \right]
}

\newcommand{\columnVector}[1]{%
  \left[
    \begin{array}{r}
    #1                           
    \end{array}
  \right]
}


\begin{document}



\author{\textsc{T. Alli$^{a}$, K. Black$^{a}$}}
\institute{$^a$Department of Mathematics, University of Georgia, GA}
\subject{Linear Algebra}
\keywords{Linear Transformation, Vectors, Matrices, Linear Algebra}

%\lecture{Partial Fractions}{partial-fractions}
%\section{Rational Functions}

\title{5.1B - Eigenvalues and Eigenvectors}
\subtitle{Directions That Do Not Change}


\date{} % {\today}

\begin{frame}
  \titlepage
\end{frame}

\begin{frame}{Outline}
  \tableofcontents
\end{frame}


\section{Goals}

\begin{frame}{Goals}

  \begin{itemize}
  \item Determine the eigenvalues and eigenvectors of a matrix.
  \item Check whether or not a vector is an eigenvalue, or whether or
    not a scalar value is an eigenvalue.
  \item Determine the eigenspace of a matrix.
  %\item Determine the basis for the eigenspace of a matrix.
  \end{itemize}

\end{frame}

\section{Definition Of Eigenvectors and EigenValues}

\begin{frame}{Eigenvectors and Eigenvalues}

  Assume that $A$ is an $n\times n$ matrix.

  \begin{block}{Definition: Eigenvector}
    An eigenvector, $\vec{v}$, of the matrix A is a vector that
    satisfies $A\vec{v}=\lambda\vec{v}$ where $\lambda$ is a scalar value.
  \end{block}

  \begin{block}{Definition: Eigenvalue}
    An eigenvalue, $\lambda$, of the matrix A is a scalar value that
    satisfies $A\vec{v}=\lambda\vec{v}$ where $\vec{v}$ is a non-zero
    vector.
  \end{block}

\end{frame}

\section{Examples}

\begin{frame}
  \frametitle{Example}

  \begin{eqnarray*}
    A & = & \twoByTwo{-32}{-36}{30}{34}
  \end{eqnarray*}

  \begin{columns}
    \column{0.5\textwidth}
    \begin{eqnarray*}
      \twoByTwo{-32}{-36}{30}{34}\columnVector{1\\-1} & = & \uncover<2->{\columnVector{4\\-4}}
    \end{eqnarray*}

    \column{0.5\textwidth}
    \begin{eqnarray*}
      \twoByTwo{-32}{-36}{30}{34}\columnVector{-6\\5} & = & \uncover<3->{\columnVector{12\\-10}}
    \end{eqnarray*}

  \end{columns}

  \vspace*{10em}
  
\end{frame}


\begin{frame}{Example}

  \begin{eqnarray*}
    A & = & \twoByTwo{6}{6}{-3}{-3}
  \end{eqnarray*}

  \begin{columns}
    \column{0.5\textwidth}
    \begin{eqnarray*}
      \twoByTwo{6}{6}{-3}{-3}\columnVector{2\\-1} & = & \uncover<2->{\columnVector{6\\-3}}
    \end{eqnarray*}

    \column{0.5\textwidth}
    \begin{eqnarray*}
      \twoByTwo{6}{6}{-3}{-3}\columnVector{-1\\1} & = & \uncover<3->{\columnVector{0\\0}}
    \end{eqnarray*}

 \end{columns}

 \vspace*{10em}
  
\end{frame}

\begin{frame}{Eigenvectors with distinct eigenvalues are linearly
    independent}

  Suppose ${\vec{v}_1,\vec{v}_2,\vec{v}_3,\ldots,\ldots,\vec{v}_j},\ldots,\vec{v}_n$ are
  eigenvectors of a matrix, $A$, and their associated eigenvalues are
  all different from one another. Assume they are \textbf{not} linearly
  independent.

  \dotfield{60}{24}
  
\end{frame}

\begin{frame}{Number of Eigenvalues for an $n\times n$ Matrix.}

  If a square matrix is $n\times n$, then there at most $n$ linearly
  independent vectors in any set of vectors.
  
\end{frame}

\begin{frame}{How To Determine The Eigenvectors?}

  First, how many eigenvectors are there for a given eigenvalue?

  Suppose $A\vec{v}=\lambda\vec{v}$.

  \dotfield{60}{20}
  
\end{frame}

\begin{frame}{How To Determine The Eigenvectors?}

  Suppose $A\vec{v}=\lambda\vec{v}$.

  \dotfield{60}{20}
  
\end{frame}

\begin{frame}{How To Determine The Eigenvectors?}

  \begin{eqnarray*}
    A & = & \twoByTwo{0.8}{0.3}{0.2}{0.7}
  \end{eqnarray*}


  \only<1>{\dotfield{60}{20}}
  \only<2->{\dotfield{60}{20}}
  
\end{frame}

\begin{frame}{How To Determine The Eigenvectors?}

  \begin{eqnarray*}
    A & = & \twoByTwo{6}{6}{-3}{-3}
  \end{eqnarray*}


  \only<1>{\dotfield{60}{20}}
  \only<2->{\dotfield{60}{20}}
  
\end{frame}

\begin{frame}{How To Determine The Eigenvectors?}

  \begin{eqnarray*}
    A & = & \threeByThree{7}{-18}{9}{3}{-8}{3}{0}{0}{-2}
  \end{eqnarray*}


  \only<1>{\dotfield{60}{20}}
  \only<2>{\dotfield{60}{20}}
  \only<3>{\dotfield{60}{20}}
  \only<4->{\dotfield{60}{20}}
  
\end{frame}

\begin{frame}{What If Zero Is An Eigenvalue?}

  \dotfield{60}{24}
  
\end{frame}


\begin{frame}{Blank Page}
  \dotfield{60}{24}
\end{frame}


\end{document}
