\documentclass[svgnames,table,,aspectratio=169]{beamer}
%\documentclass[svgnames,table,handout,aspectratio=129]{beamer}
\usepackage{hhline}
\usepackage{etoolbox}
\usepackage{tikz}
\usepackage{mathtools}
\usepackage{amssymb}
%\usepackage{/usr/lib64/R/share/texmf/Sweave}
\usepackage{polynom}
\usepackage{qrcode}


%\input{latexdefinitions}
\definecolor{georgiaRed}{RGB}{100,0,00}
\definecolor{mediumGray}{gray}{0.6}



\usetheme{Frankfurt}%
%\usetheme{Warsaw}%
%\useoutertheme{smoothbars}


%\usecolortheme{seagull}
\usecolortheme{beaver}
\logo{\includegraphics[height=.125in]{ugaLogo}}

% Note that the colour definitions are given in the latexDefinitions
% file.
\setbeamercolor{palette primary}{fg=georgiaRed,bg=white}
\setbeamercolor{palette secondary}{fg=georgiaRed,bg=white}
\setbeamercolor{palette tertiary}{fg=georgiaRed,bg=white}
\setbeamercolor{palette quaternary}{bg=mediumGray,fg=black}
\setbeamercolor{block title}{fg=white,bg=georgiaRed}
\setbeamercolor{block body}{fg=black,bg=black!10}
\setbeamercolor{titlelike}{bg=georgiaRed,fg=white} % parent=palette quaternary}

% Define the variable to determine whether or not the clicker quizzes
% are visible in the resulting output.
\newtoggle{clicker}
\toggletrue{clicker}
%\togglefalse{clicker}


% To display a lecture uncomment out the "includeonly" line below to
% match the name of the file. You do not have to do anything with the
% lecture line below and can leave it commented out. It is in place
% because at one time we had multiple lectures within a file, but that
% has been changed.



\mode<presentation>{
  \setbeamercovered{invisible}
  \setbeameroption{hide notes}
}

\mode<handout>{
 
  \usepackage{pgfpages}
  %\pgfpagesuselayout{4 on 1}[letterpaper, border shrink=5mm]
  \pgfpagesuselayout{resize to}[letterpaper,border shrink=5mm]
  \setbeameroption{show notes}


  %\pgfpagesphysicalpageoptions{logical pages=2,physical
  %height=\pgfpageoptionheight,physical width=\pgfpageoptionwidth}
  % Set up the pages for notes.
  % This idea and some code came from
  % http://www.guidodiepen.nl/2009/07/creating-latex-beamer-handouts-with-notes/



  \pgfpagesdeclarelayout{3 on 1 with notes} {
    \edef\pgfpageoptionheight{11in} %\the\paperheight}
    \edef\pgfpageoptionwidth{8.5in} %\the\paperwidth}
    \edef\pgfpageoptionborder{0pt}
  }

  {

	\AtBeginDocument{
      \newbox\notesbox
      \setbox\notesbox=\vbox{
        \hsize=\paperwidth
        \vskip-2.5cm\hskip-5cm\vbox{
          \textcolor{light-gray}{\hrule width 12.6cm\vskip0.5cm}
          \textcolor{light-gray}{\hrule width 12.6cm\vskip0.5cm}
          \textcolor{light-gray}{\hrule width 12.6cm\vskip0.5cm}
          \textcolor{light-gray}{\hrule width 12.6cm\vskip0.5cm}
          \textcolor{light-gray}{\hrule width 12.6cm\vskip0.5cm}
          \textcolor{light-gray}{\hrule width 12.6cm\vskip0.5cm}
          \textcolor{light-gray}{\hrule width 12.6cm\vskip0.5cm}
          \textcolor{light-gray}{\hrule width 12.6cm\vskip0.5cm}
          \textcolor{light-gray}{\hrule width 12.6cm\vskip0.5cm}
          \textcolor{light-gray}{\hrule width 12.6cm\vskip0.5cm}
          \textcolor{light-gray}{\hrule width 12.6cm\vskip0.5cm}
          \textcolor{light-gray}{\hrule width 12.6cm\vskip0.5cm}
          \textcolor{light-gray}{\hrule width 12.6cm\vskip0.5cm}
          \textcolor{light-gray}{\hrule width 12.6cm\vskip0.5cm}
          \textcolor{light-gray}{\hrule width 12.6cm\vskip0.5cm}
          \textcolor{light-gray}{\hrule width 12.6cm\vskip0.5cm}
          \textcolor{light-gray}{\hrule width 12.6cm\vskip0.5cm}
          \textcolor{light-gray}{\hrule width 12.6cm\vskip0.5cm}
          \textcolor{light-gray}{\hrule width 12.6cm\vskip0.5cm}

          \vspace*{-9.75cm}
          \textcolor{light-gray}{\rule[-1.0cm]{1pt}{9.25cm}\hskip0.5cm}
          \textcolor{light-gray}{\rule[-1.0cm]{1pt}{9.25cm}\hskip0.5cm}
          \textcolor{light-gray}{\rule[-1.0cm]{1pt}{9.25cm}\hskip0.5cm}
          \textcolor{light-gray}{\rule[-1.0cm]{1pt}{9.25cm}\hskip0.5cm}
          \textcolor{light-gray}{\rule[-1.0cm]{1pt}{9.25cm}\hskip0.5cm}
          \textcolor{light-gray}{\rule[-1.0cm]{1pt}{9.25cm}\hskip0.5cm}
          \textcolor{light-gray}{\rule[-1.0cm]{1pt}{9.25cm}\hskip0.5cm}
          \textcolor{light-gray}{\rule[-1.0cm]{1pt}{9.25cm}\hskip0.5cm}
          \textcolor{light-gray}{\rule[-1.0cm]{1pt}{9.25cm}\hskip0.5cm}
          \textcolor{light-gray}{\rule[-1.0cm]{1pt}{9.25cm}\hskip0.5cm}
          \textcolor{light-gray}{\rule[-1.0cm]{1pt}{9.25cm}\hskip0.5cm}
          \textcolor{light-gray}{\rule[-1.0cm]{1pt}{9.25cm}\hskip0.5cm}
          \textcolor{light-gray}{\rule[-1.0cm]{1pt}{9.25cm}\hskip0.5cm}
          \textcolor{light-gray}{\rule[-1.0cm]{1pt}{9.25cm}\hskip0.5cm}
          \textcolor{light-gray}{\rule[-1.0cm]{1pt}{9.25cm}\hskip0.5cm}
          \textcolor{light-gray}{\rule[-1.0cm]{1pt}{9.25cm}\hskip0.5cm}
          \textcolor{light-gray}{\rule[-1.0cm]{1pt}{9.25cm}\hskip0.5cm}
          \textcolor{light-gray}{\rule[-1.0cm]{1pt}{9.25cm}\hskip0.5cm}
          \textcolor{light-gray}{\rule[-1.0cm]{1pt}{9.25cm}\hskip0.5cm}
          \textcolor{light-gray}{\rule[-1.0cm]{1pt}{9.25cm}\hskip0.5cm}

        }

      }

    \pgfpagesphysicalpageoptions
    {%
      logical pages=6,%
      physical height=\pgfpageoptionheight,%
      physical width=\pgfpageoptionwidth,%
      last logical shipout=3%
    }
    
    \pgfpageslogicalpageoptions{1}
    {%
      border shrink=\pgfpageoptionborder,%
      resized width=.5\pgfphysicalwidth,%
      resized height=.33\pgfphysicalheight,%
      center=\pgfpoint{.25\pgfphysicalwidth}{.82\pgfphysicalheight}%
    }%
    \pgfpageslogicalpageoptions{2}
    {%
      border shrink=\pgfpageoptionborder,%
      resized width=.5\pgfphysicalwidth,%
      resized height=.33\pgfphysicalheight,%
      center=\pgfpoint{.25\pgfphysicalwidth}{.47\pgfphysicalheight}%
    }%
    \pgfpageslogicalpageoptions{3}
    {%
      border shrink=\pgfpageoptionborder,%
      resized width=.5\pgfphysicalwidth,%
      resized height=.33\pgfphysicalheight,%
      center=\pgfpoint{.25\pgfphysicalwidth}{.17\pgfphysicalheight}%
    }%	
	\pgfpageslogicalpageoptions{4}
    {%
      border shrink=\pgfpageoptionborder,%
      resized width=.5\pgfphysicalwidth,%
      resized height=.33\pgfphysicalheight,%
      center=\pgfpoint{.85\pgfphysicalwidth}{.82\pgfphysicalheight},%
      copy from=4
    }%
    \pgfpageslogicalpageoptions{5}
    {%
      border shrink=\pgfpageoptionborder,%
      resized width=.5\pgfphysicalwidth,%
      resized height=.33\pgfphysicalheight,%
      center=\pgfpoint{.85\pgfphysicalwidth}{.47\pgfphysicalheight},%
      copy from=5
    }%
    \pgfpageslogicalpageoptions{6}
    {%
      border shrink=\pgfpageoptionborder,%
      resized width=.5\pgfphysicalwidth,%
      resized height=.33\pgfphysicalheight,%
      center=\pgfpoint{.85\pgfphysicalwidth}{.17\pgfphysicalheight},%
      copy from=6
    }%
    
      \pgfpagesshipoutlogicalpage{4}\copy\notesbox
      \pgfpagesshipoutlogicalpage{5}\copy\notesbox
      \pgfpagesshipoutlogicalpage{6}\copy\notesbox
    }
  }

  \pgfpagesuselayout{3 on 1 with notes}

}

\setbeamercolor{upper separation line head}{bg=red}
\setbeamercolor{headline}{bg=red}
\setbeamertemplate{headline}
{%
\begin{beamercolorbox}{section in head/foot}
\insertsectionnavigationhorizontal{.75\textwidth}{}{}
\hfill \insertpagenumber /\insertdocumentendpage
\end{beamercolorbox}%
}
\setbeamercolor{section number projected}{bg=red,fg=black}
\setbeamercolor{subsection number projected}{bg=red,fg=black}
%\setbeamercolor{frametitle}{bg=lightgray,fg=black}

\setbeamertemplate{itemize item}{\color{georgiaRed}$\blacklozenge$}
\setbeamertemplate{itemize subitem}{\color{georgiaRed}$\blacktriangleright$}

\newcommand{\dotfield}[2]{%
  \begin{tikzpicture}[y=0.25cm, x=0.25cm,font=\sffamily]
    \foreach \y in {0,...,#2} {
      \foreach \x in {0,...,#1} {
        \draw[fill=georgiaRed,opacity=0.1] (\x,\y)  circle [radius=0.03em];
      }
    }
  \end{tikzpicture}
}

\newcommand{\twoByTwo}[4]{%
  \left[
    \begin{array}{rr}
      #1 & #2 \\
      #3 & #4 \\
    \end{array}
  \right]
}

\newcommand{\threeByThree}[9]{%
  \left[
    \begin{array}{rrr}
      #1 & #2 & #3 \\
      #4 & #5 & #6 \\
      #7 & #8 & #9
    \end{array}
  \right]
}

\newcommand{\columnVector}[1]{%
  \left[
    \begin{array}{r}
    #1                           
    \end{array}
  \right]
}


\begin{document}



\author{\textsc{T. Alli$^{a}$, K. Black$^{a}$}}
\institute{$^a$Department of Mathematics, University of Georgia, GA}
\subject{Linear Algebra}
\keywords{Linear Transformation, Vectors, Matrices, Linear Algebra}

%\lecture{Partial Fractions}{partial-fractions}
%\section{Rational Functions}

\title{Linear Least Squares Approximation Of Coefficients}
\subtitle{Which Plane Is ``Best?''}


\date{} % {\today}

\begin{frame}
  \titlepage
\end{frame}

\begin{frame}{Outline}
  \tableofcontents
\end{frame}


\section{The Idea}

\begin{frame}{The Idea}

  \begin{itemize}
  \item We have measurable quantities.
  \item We assume a linear relationship between them.
  \item We want to approximate the slopes and the intercept.
  \item We conduct an experiment to measure the quantities.
  \item We make assumptions about the experiment.
    \begin{itemize}
    \item The measurements are independent of one another.
    \item The dependent variables are changed in a reliable, repeatable, and
      accurately measurable way.
    \item For each set of values for the independent variable a
      measurement of an outcome (a response) is made.
    \item The biggest source of error is in measuring the response.
    \item The error of the measurement is ``nice.''
    \end{itemize}
  \item We have more measurements than unknown values.
  \end{itemize}

\end{frame}

\section{Linear Mathematical Model}

\begin{frame}{Linear Mathematical Model}

  \begin{itemize}
  \item We have independent variables, $x_1$, $x_2$, $x_3$, $\ldots$,
    $x_l$.
  \item We have a dependent variable, $z$.
  \item The value of $z$ can be calculated by a linear relationship,
    \begin{eqnarray*}
      z & = & m_1 x_1 + m_2 x_2 + m_3 x_3 + \cdots + m_l x_l + b.
    \end{eqnarray*}
  \item The values of $m_1$, $m_2$, $m_3$, $\ldots$, $m_l$, and $b$  are constants.
  \end{itemize}

  \vfill

  \uncover<2->{Question: How do we approximate the values of these
    constants for a given physical system?}

  \vfill

\end{frame}

\section{Experimental Data}

\begin{frame}{Experimental Data}

  \begin{itemize}
  \item Determine values of $x_1$, $x_2$, $x_3$, $\ldots$, $x_l$ that
    can be replicated and provide the ``best'' approximation. \\
    (Class Advertisement: Design of Experiments.)
  \item Perform $n$ experiments with the different values (many
    repeated),
    \begin{eqnarray*}
      \begin{array}{lr|r|r|r|r|r}
        \mathrm{Experiment~1}: & x_{11}  & x_{21} & x_{31} & \ldots & x_{l1} & z_1 \\
        \mathrm{Experiment~2}: & x_{12}  & x_{22} & x_{32} & \ldots & x_{l2} & z_2 \\
        \mathrm{Experiment~3}: & x_{13}  & x_{23} & x_{33} & \ldots & x_{l3} & z_3 \\
        \vdots & \vdots & \vdots & \vdots & \vdots & \vdots \\
        \mathrm{Experiment~n}: & x_{1n}  & x_{2n} & x_{3n} & \ldots & x_{l1} & z_n 
      \end{array}
    \end{eqnarray*}
  \end{itemize}
  
\end{frame}

\begin{frame}{Mathematical Model With Error}

  \begin{itemize}
  \item Our original assumption:
    \begin{eqnarray*}
      z_j & = & m_1 x_{1j} + m_2 x_{2j} + m_3 x_{3j} + \cdots + m_l x_{lj} + b.
    \end{eqnarray*}

  \item The world is a messy place:
    \begin{eqnarray*}
      z_j & = & m_1 x_{1j} + m_2 x_{2j} + m_3 x_{3j} + \cdots + m_l x_{lj} + b + \epsilon_j.
    \end{eqnarray*}

  \item So we can calculate the error
    \begin{eqnarray*}
      \epsilon_j  & = & z_j - \left(m_1 x_{1j} + m_2 x_{2j} + m_3 x_{3j} + \cdots + m_l x_{lj} + b \right).
    \end{eqnarray*}

  \end{itemize}
  
\end{frame}

\begin{frame}{Assume The Experiment Was Consistent}

  Assumptions about the error term, $\epsilon_j$.
  \begin{itemize}
  \item Independent of the other measurements.
  \item Normally Distributed.
  \item There is no bias (mean is zero)
  \item Same variation for each measurement.
  \item Jargon: normally distributed, i.i.d. 
  \end{itemize}

\end{frame}

\section{Probability Associated With The Error}

\begin{frame}{Probability Associated With The Error}

  \begin{itemize}
  \item We assume that the probability we obtained the measurement
    associated with each measurement is proportional to a normal
    probability distribution function,
    \begin{eqnarray*}
      \sigma_j & \propto & e^{-\left(z_j - \left(m_1 x_{1j} + m_2 x_{2j} + m_3 x_{3j} + \cdots + m_l x_{lj} + b\right)\right)^2/(2\sigma^2)}
    \end{eqnarray*}
  \item Because they are independent we can approximate the
    probability associated with all of the errors by multiplying them
    together.
    \begin{eqnarray*}
      \mathrm{Prob.} & \propto & e^{-\left(z_1 - \left(m_1 x_{11} + m_2 x_{21} + m_3 x_{31} + \cdots + m_l x_{l1} + b\right)\right)^2/(2\sigma^2)} \\
      & & e^{-\left(z_2 - \left(m_1 x_{12} + m_2 x_{22} + m_3 x_{32} + \cdots + m_l x_{l2} + b\right)\right)^2/(2\sigma^2)} \\
      & & e^{-\left(z_3 - \left(m_1 x_{13} + m_2 x_{23} + m_3 x_{33} + \cdots + m_l x_{l3} + b\right)\right)^2/(2\sigma^2)} \\
      & & \vdots \\
      & & e^{-\left(z_n - \left(m_1 x_{1n} + m_2 x_{2n} + m_3 x_{3n} + \cdots + m_l x_{ln} + b\right)\right)^2/(2\sigma^2)} 
    \end{eqnarray*}
    
  \end{itemize}
  
\end{frame}

\begin{frame}{Maximize the Probability}

  We calculate the values of $m_1$, $m_2$, $m_3$, $\ldots$, $m_l$, and
  $b$ that will maximize the probability that we would have obtained the given results:
  \begin{eqnarray*}
    \frac{\partial}{\partial m_1} \mathrm{Prob.} & = & 0, \\
    \frac{\partial}{\partial m_2} \mathrm{Prob.} & = & 0, \\
    \frac{\partial}{\partial m_3} \mathrm{Prob.} & = & 0, \\
    \vdots & & \\
    \frac{\partial}{\partial m_l} \mathrm{Prob.} & = & 0, \\
    \frac{\partial}{\partial b} \mathrm{Prob.} & = & 0.
  \end{eqnarray*}
  
\end{frame}

\begin{frame}{Side Note}
  \label{slide:minimization}

  This is the same as maximizing the logarithm of the probability
  (which is a heck of a lot easier to work with and easier to make
  sense out of).

  \begin{eqnarray*}
    \ln\left(\mathrm{Prob.}\right)
    & \propto & -\frac{1}{2}\left(z_1 - \left(m_1 x_{11} + m_2 x_{21} + m_3 x_{31} + \cdots + m_l x_{l1} + b\right)\right)^2 \\
      & & -\frac{1}{2}\left(z_2 - \left(m_1 x_{12} + m_2 x_{22} + m_3 x_{32} + \cdots + m_l x_{l2} + b\right)\right)^2 \\
      & & -\frac{1}{2}\left(z_3 - \left(m_1 x_{13} + m_2 x_{23} + m_3 x_{33} + \cdots + m_l x_{l3} + b\right)\right)^2 \\
      & & \vdots \\
      & & -\frac{1}{2}\left(z_n - \left(m_1 x_{1n} + m_2 x_{2n} + m_3 x_{3n} + \cdots + m_l x_{ln} + b\right)\right)^2 
  \end{eqnarray*}

  
  
\end{frame}

\begin{frame}{Geometric Interpretation}
  \dotfield{60}{24}
\end{frame}

\section{The System Of Equations}

\begin{frame}{The System Of Equations}

  If you take the partial derivatives shown on slide
  \ref{slide:minimization}, then you get a system of equations of the
  form
  \begin{eqnarray*}
    \sum_{j=1}^n x_{1j} \left( z_j  - m_1 x_{1j} - m_2 x_{2j} - \cdots - m_lx_{lj} - b\right) & = & 0, \\
    \sum_{j=1}^n x_{2j} \left( z_j  - m_1 x_{1j} - m_2 x_{2j} - \cdots - m_lx_{lj} - b \right) & = & 0, \\
    \vdots & & \\
    \sum_{j=1}^n x_{nj} \left( z_j  - m_1 x_{1j} - m_2 x_{2j} - \cdots - m_lx_{lj} - b \right) & = & 0, \\
    \sum_{j=1}^n z_j - m_1 x_{1j} - m_2 x_{2j} - \cdots - m_lx_{lj} - b & = & 0. \\
  \end{eqnarray*}
  
\end{frame}

\begin{frame}{Yuk}

  What the....
  
\end{frame}

\begin{frame}{Another Way}

  Ideally....
  \begin{eqnarray*}
    m_1 x_{11} + m_2 x_{21} + \cdots + m_lx_{l1} + b  & = &   z_1 \\
    m_1 x_{12} + m_2 x_{22} + \cdots + m_lx_{l2} + b  & = &   z_2 \\
    \vdots & & \\
    m_1 x_{1n} + m_2 x_{2n} + \cdots + m_lx_{ln} + b & = & z_n  \\
  \end{eqnarray*}

  \begin{eqnarray*}
    \left[
    \begin{array}{rrrrrr}
      x_{11} & x_{21} & x_{31} &  \cdots & x_{l1} & 1 \\
      x_{12} & x_{22} & x_{32} &  \cdots & x_{l2} & 1 \\
      x_{13} & x_{23} & x_{33} &  \cdots & x_{l3} & 1 \\

      x_{1n} & x_{2n} & x_{3n} &  \cdots & x_{ln} & 1 
    \end{array}
    \right]
    \left[
    \begin{array}{r}
      m_1 \\
      m_2 \\
      m_3 \\
      \vdots \\
      m_l \\
      b  
    \end{array}
    \right]
             & = &
                   \left[
                   \begin{array}{r}
                     z_1 \\
                     z_2 \\
                     z_3 \\
                     \vdots \\
                     z_l
                   \end{array}
    \right]
  \end{eqnarray*}
  
  
\end{frame}

\begin{frame}{Another Way}

  Ideally....
  \begin{eqnarray*}
    \left[
    \begin{array}{rrrrrr}
      x_{11} & x_{21} & x_{31} &  \cdots & x_{l1} & 1 \\
      x_{12} & x_{22} & x_{32} &  \cdots & x_{l2} & 1 \\
      x_{13} & x_{23} & x_{33} &  \cdots & x_{l3} & 1 \\

      x_{1n} & x_{2n} & x_{3n} &  \cdots & x_{ln} & 1 
    \end{array}
    \right]
    \left[
    \begin{array}{r}
      m_1 \\
      m_2 \\
      m_3 \\
      \vdots \\
      m_l \\
      b  
    \end{array}
    \right]
             & ``=" &
                   \left[
                   \begin{array}{r}
                     z_1 \\
                     z_2 \\
                     z_3 \\
                     \vdots \\
                     z_l
                   \end{array}
    \right]
  \end{eqnarray*}

  Or
  \begin{eqnarray*}
    A \vec{p} & ``=" & \vec{z}
  \end{eqnarray*}

  See next video on how to proceed....
  
  
\end{frame}




\end{document}
