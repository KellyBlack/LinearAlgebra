\documentclass[svgnames,table,,aspectratio=169]{beamer}
%\documentclass[svgnames,table,handout,aspectratio=129]{beamer}
\usepackage{hhline}
\usepackage{etoolbox}
\usepackage{tikz}
\usepackage{mathtools}
\usepackage{amssymb}
%\usepackage{/usr/lib64/R/share/texmf/Sweave}
\usepackage{polynom}
\usepackage{qrcode}


%\input{latexdefinitions}
\definecolor{georgiaRed}{RGB}{100,0,00}
\definecolor{mediumGray}{gray}{0.6}



\usetheme{Frankfurt}%
%\usetheme{Warsaw}%
%\useoutertheme{smoothbars}


%\usecolortheme{seagull}
\usecolortheme{beaver}
\logo{\includegraphics[height=.125in]{ugaLogo}}

% Note that the colour definitions are given in the latexDefinitions
% file.
\setbeamercolor{palette primary}{fg=georgiaRed,bg=white}
\setbeamercolor{palette secondary}{fg=georgiaRed,bg=white}
\setbeamercolor{palette tertiary}{fg=georgiaRed,bg=white}
\setbeamercolor{palette quaternary}{bg=mediumGray,fg=black}
\setbeamercolor{block title}{fg=white,bg=georgiaRed}
\setbeamercolor{block body}{fg=black,bg=black!10}
\setbeamercolor{titlelike}{bg=georgiaRed,fg=white} % parent=palette quaternary}

% Define the variable to determine whether or not the clicker quizzes
% are visible in the resulting output.
\newtoggle{clicker}
\toggletrue{clicker}
%\togglefalse{clicker}


% To display a lecture uncomment out the "includeonly" line below to
% match the name of the file. You do not have to do anything with the
% lecture line below and can leave it commented out. It is in place
% because at one time we had multiple lectures within a file, but that
% has been changed.



\mode<presentation>{
  \setbeamercovered{invisible}
  \setbeameroption{hide notes}
}

\mode<handout>{
 
  \usepackage{pgfpages}
  %\pgfpagesuselayout{4 on 1}[letterpaper, border shrink=5mm]
  \pgfpagesuselayout{resize to}[letterpaper,border shrink=5mm]
  \setbeameroption{show notes}


  %\pgfpagesphysicalpageoptions{logical pages=2,physical
  %height=\pgfpageoptionheight,physical width=\pgfpageoptionwidth}
  % Set up the pages for notes.
  % This idea and some code came from
  % http://www.guidodiepen.nl/2009/07/creating-latex-beamer-handouts-with-notes/



  \pgfpagesdeclarelayout{3 on 1 with notes} {
    \edef\pgfpageoptionheight{11in} %\the\paperheight}
    \edef\pgfpageoptionwidth{8.5in} %\the\paperwidth}
    \edef\pgfpageoptionborder{0pt}
  }

  {

	\AtBeginDocument{
      \newbox\notesbox
      \setbox\notesbox=\vbox{
        \hsize=\paperwidth
        \vskip-2.5cm\hskip-5cm\vbox{
          \textcolor{light-gray}{\hrule width 12.6cm\vskip0.5cm}
          \textcolor{light-gray}{\hrule width 12.6cm\vskip0.5cm}
          \textcolor{light-gray}{\hrule width 12.6cm\vskip0.5cm}
          \textcolor{light-gray}{\hrule width 12.6cm\vskip0.5cm}
          \textcolor{light-gray}{\hrule width 12.6cm\vskip0.5cm}
          \textcolor{light-gray}{\hrule width 12.6cm\vskip0.5cm}
          \textcolor{light-gray}{\hrule width 12.6cm\vskip0.5cm}
          \textcolor{light-gray}{\hrule width 12.6cm\vskip0.5cm}
          \textcolor{light-gray}{\hrule width 12.6cm\vskip0.5cm}
          \textcolor{light-gray}{\hrule width 12.6cm\vskip0.5cm}
          \textcolor{light-gray}{\hrule width 12.6cm\vskip0.5cm}
          \textcolor{light-gray}{\hrule width 12.6cm\vskip0.5cm}
          \textcolor{light-gray}{\hrule width 12.6cm\vskip0.5cm}
          \textcolor{light-gray}{\hrule width 12.6cm\vskip0.5cm}
          \textcolor{light-gray}{\hrule width 12.6cm\vskip0.5cm}
          \textcolor{light-gray}{\hrule width 12.6cm\vskip0.5cm}
          \textcolor{light-gray}{\hrule width 12.6cm\vskip0.5cm}
          \textcolor{light-gray}{\hrule width 12.6cm\vskip0.5cm}
          \textcolor{light-gray}{\hrule width 12.6cm\vskip0.5cm}

          \vspace*{-9.75cm}
          \textcolor{light-gray}{\rule[-1.0cm]{1pt}{9.25cm}\hskip0.5cm}
          \textcolor{light-gray}{\rule[-1.0cm]{1pt}{9.25cm}\hskip0.5cm}
          \textcolor{light-gray}{\rule[-1.0cm]{1pt}{9.25cm}\hskip0.5cm}
          \textcolor{light-gray}{\rule[-1.0cm]{1pt}{9.25cm}\hskip0.5cm}
          \textcolor{light-gray}{\rule[-1.0cm]{1pt}{9.25cm}\hskip0.5cm}
          \textcolor{light-gray}{\rule[-1.0cm]{1pt}{9.25cm}\hskip0.5cm}
          \textcolor{light-gray}{\rule[-1.0cm]{1pt}{9.25cm}\hskip0.5cm}
          \textcolor{light-gray}{\rule[-1.0cm]{1pt}{9.25cm}\hskip0.5cm}
          \textcolor{light-gray}{\rule[-1.0cm]{1pt}{9.25cm}\hskip0.5cm}
          \textcolor{light-gray}{\rule[-1.0cm]{1pt}{9.25cm}\hskip0.5cm}
          \textcolor{light-gray}{\rule[-1.0cm]{1pt}{9.25cm}\hskip0.5cm}
          \textcolor{light-gray}{\rule[-1.0cm]{1pt}{9.25cm}\hskip0.5cm}
          \textcolor{light-gray}{\rule[-1.0cm]{1pt}{9.25cm}\hskip0.5cm}
          \textcolor{light-gray}{\rule[-1.0cm]{1pt}{9.25cm}\hskip0.5cm}
          \textcolor{light-gray}{\rule[-1.0cm]{1pt}{9.25cm}\hskip0.5cm}
          \textcolor{light-gray}{\rule[-1.0cm]{1pt}{9.25cm}\hskip0.5cm}
          \textcolor{light-gray}{\rule[-1.0cm]{1pt}{9.25cm}\hskip0.5cm}
          \textcolor{light-gray}{\rule[-1.0cm]{1pt}{9.25cm}\hskip0.5cm}
          \textcolor{light-gray}{\rule[-1.0cm]{1pt}{9.25cm}\hskip0.5cm}
          \textcolor{light-gray}{\rule[-1.0cm]{1pt}{9.25cm}\hskip0.5cm}

        }

      }

    \pgfpagesphysicalpageoptions
    {%
      logical pages=6,%
      physical height=\pgfpageoptionheight,%
      physical width=\pgfpageoptionwidth,%
      last logical shipout=3%
    }
    
    \pgfpageslogicalpageoptions{1}
    {%
      border shrink=\pgfpageoptionborder,%
      resized width=.5\pgfphysicalwidth,%
      resized height=.33\pgfphysicalheight,%
      center=\pgfpoint{.25\pgfphysicalwidth}{.82\pgfphysicalheight}%
    }%
    \pgfpageslogicalpageoptions{2}
    {%
      border shrink=\pgfpageoptionborder,%
      resized width=.5\pgfphysicalwidth,%
      resized height=.33\pgfphysicalheight,%
      center=\pgfpoint{.25\pgfphysicalwidth}{.47\pgfphysicalheight}%
    }%
    \pgfpageslogicalpageoptions{3}
    {%
      border shrink=\pgfpageoptionborder,%
      resized width=.5\pgfphysicalwidth,%
      resized height=.33\pgfphysicalheight,%
      center=\pgfpoint{.25\pgfphysicalwidth}{.17\pgfphysicalheight}%
    }%	
	\pgfpageslogicalpageoptions{4}
    {%
      border shrink=\pgfpageoptionborder,%
      resized width=.5\pgfphysicalwidth,%
      resized height=.33\pgfphysicalheight,%
      center=\pgfpoint{.85\pgfphysicalwidth}{.82\pgfphysicalheight},%
      copy from=4
    }%
    \pgfpageslogicalpageoptions{5}
    {%
      border shrink=\pgfpageoptionborder,%
      resized width=.5\pgfphysicalwidth,%
      resized height=.33\pgfphysicalheight,%
      center=\pgfpoint{.85\pgfphysicalwidth}{.47\pgfphysicalheight},%
      copy from=5
    }%
    \pgfpageslogicalpageoptions{6}
    {%
      border shrink=\pgfpageoptionborder,%
      resized width=.5\pgfphysicalwidth,%
      resized height=.33\pgfphysicalheight,%
      center=\pgfpoint{.85\pgfphysicalwidth}{.17\pgfphysicalheight},%
      copy from=6
    }%
    
      \pgfpagesshipoutlogicalpage{4}\copy\notesbox
      \pgfpagesshipoutlogicalpage{5}\copy\notesbox
      \pgfpagesshipoutlogicalpage{6}\copy\notesbox
    }
  }

  \pgfpagesuselayout{3 on 1 with notes}

}

\setbeamercolor{upper separation line head}{bg=red}
\setbeamercolor{headline}{bg=red}
\setbeamertemplate{headline}
{%
\begin{beamercolorbox}{section in head/foot}
\insertsectionnavigationhorizontal{.75\textwidth}{}{}
\hfill \insertpagenumber /\insertdocumentendpage
\end{beamercolorbox}%
}
\setbeamercolor{section number projected}{bg=red,fg=black}
\setbeamercolor{subsection number projected}{bg=red,fg=black}
%\setbeamercolor{frametitle}{bg=lightgray,fg=black}

\setbeamertemplate{itemize item}{\color{georgiaRed}$\blacklozenge$}
\setbeamertemplate{itemize subitem}{\color{georgiaRed}$\blacktriangleright$}

\newcommand{\dotfield}[2]{%
  \begin{tikzpicture}[y=0.25cm, x=0.25cm,font=\sffamily]
    \foreach \y in {0,...,#2} {
      \foreach \x in {0,...,#1} {
        \draw[fill=georgiaRed,opacity=0.1] (\x,\y)  circle [radius=0.03em];
      }
    }
  \end{tikzpicture}
}

\newcommand{\twoByTwo}[4]{%
  \left[
    \begin{array}{rr}
      #1 & #2 \\
      #3 & #4 \\
    \end{array}
  \right]
}

\newcommand{\threeByThree}[9]{%
  \left[
    \begin{array}{rrr}
      #1 & #2 & #3 \\
      #4 & #5 & #6 \\
      #7 & #8 & #9
    \end{array}
  \right]
}

\newcommand{\columnVector}[1]{%
  \left[
    \begin{array}{r}
    #1                           
    \end{array}
  \right]
}


\begin{document}



\author{\textsc{T. Alli$^{a}$, K. Black$^{a}$}}
\institute{$^a$Department of Mathematics, University of Georgia, GA}
\subject{Linear Algebra}
\keywords{Linear Transformation, Vectors, Matrices, Linear Algebra}

%\lecture{Partial Fractions}{partial-fractions}
%\section{Rational Functions}

\title{Section 3.6: The Invertible Matrix Theorem}
\subtitle{When Does The Inverse Exist?}


\date{} % {\today}

\begin{frame}
  \titlepage
\end{frame}

\begin{frame}{Outline}
  \tableofcontents
\end{frame}


\section{Goals}

\begin{frame}{Goals}

  \begin{itemize}
  \item Determine the product of two matrices.
  \item Use the product of two matrices to represent the composition
    of two linear transformations.
  \item Determine when the inverse of a matrix exists.
  \item Use the relationship between the column space of a matrix and
    the inverse to determine the nature of the solutions to a linear
    system.
  \end{itemize}

\end{frame}

\section{Preliminaries: Compositions Of Linear Transformations}

\begin{frame}{Compositions Of Linear Transformations}

  \begin{itemize}
  \item If $A$ is an $m\times n$ matrix, and $B$ is an $n\times k$
    matrix then $AB$ is an $n\times k$ matrix.
  \item If $T:\mathbb{R}^n\rightarrow\mathbb{R}^m$ then it can be
    represented as the product of an $n\times m$ matrix.
  \item If $S:\mathbb{R}^m\rightarrow\mathbb{R}^k$ then it can be
    represented as the product of an $m\times k$ matrix.
  \end{itemize}

  Question: How do we represent $S\left(T\left(\vec{x}\right)\right)$?

  \dotfield{60}{18}
  
\end{frame}

\section{Special Case: $T:{\mathbb R}^n\rightarrow{\mathbb R}^n$}

\begin{frame}{Special Case: $T:{\mathbb R}^n\rightarrow{\mathbb R}^n$}

  Solutions to $T\left(\vec{x}\right)=\vec{b}$ (Is associated with
  matrix $A$.)
  \begin{itemize}
  \item If the transformation is one-to-one:
    \begin{itemize}
    \item Solutions are always unique if they exist.
    \item Columns of the matrix associated with $T$ must be linearly
      independent.
    \item $A$ must have $n$ pivots.
    \item $A$ must be onto since it is ${\mathbb R}^n$.
    \item $A$ must be invertible.
    \end{itemize}
  \item If the transformation is onto:
    \begin{itemize}
    \item Solutions always exist.
    \item $A$ must have $n$ pivots.
    \item $A$ must be one-to-one since it is ${\mathbb R}^n$.
    \item Columns of the matrix associated with $T$ must be linearly
      independent.
    \item $A$ must be invertible.
    \end{itemize}
  \end{itemize}

\end{frame}

\begin{frame}
  \frametitle{The Invertible Matrix Theorem}

  The following statements about $T\left(\vec{x}\right)=\vec{b}$ are
  equivalent. ($T$ associated with matrix $A$.)
  \begin{itemize}
  \item $A$ is invertible.
  \item $A$ has $n$ pivots.
  \item $\mathrm{Null}(A)=\left\{\vec{0}\right\}$.
  \item The columns of $A$ are linearly independent.
  \item The columns of $A$ span $\mathbb{R}^n$.
  \item $A\vec{x}=\vec{b}$ has a unique solution for every $\vec{b}$
    in $\mathbb{R}^n$.
  \item $T$ is invertible.
  \item $T$ is one-to-one.
  \item $T$ is onto.
  \end{itemize}

  From page 183 of the book.
  
\end{frame}

\begin{frame}{So What?}

  To show that a matrix is invertible you just need to show one of the
  properties in the previous page. If any one fails then the matrix is
  not invertible.
  
\end{frame}

\section{Other Results}

\begin{frame}{Other Results}

  \begin{itemize}
  \item If $A$ is invertible then its RREF is the identity matrix.
  \item $A\vec{x}=\vec{0}$ has only one solution if $A$ is invertible.
  \item If the columns of $A$ form $n$ linearly independent vectors
    then the matrix is invertible, and the columns form a basis for
    $\mathbb{R}^n$.
  \item Check out page 184 of the book.
  \end{itemize}
\end{frame}

\begin{frame}{The Inverse}

  Suppose a matrix $A$ has an inverse. Which inverse is it? Both!

  Suppose we have two $n\times n$ matrices where
  \begin{eqnarray*}
    AB & = & I_n.
  \end{eqnarray*}

  Is $BA=I_n$????

  \dotfield{60}{18}
  
\end{frame}

\section{Examples}

\begin{frame}{Example: Is This Matrix Invertible?}

  \begin{eqnarray*}
    A & = &
            \left[
            \begin{array}{rrr}
               1 &  8 &  7 \\
              -1 &  3 &  4 \\
               4 & -5 & -9
            \end{array}
            \right]
  \end{eqnarray*}

  \uncover<2->{
  \begin{eqnarray*}
    rref(A) & = &
            \left[
            \begin{array}{rrr}
               1 &  0 & -1 \\
               0 &  1 &  1 \\
               0 &  0 &  0
            \end{array}
            \right]
  \end{eqnarray*}
    }
  
  \end{frame}

  \begin{frame}{RREF}

    If the RREF if a square matrix is the identity matrix it is
    invertible.

    \dotfield{60}{24}
    
  \end{frame}

  \begin{frame}{Matrix Associated With The Inverse}

    Suppose the inverse of $T$ is $S$.
    \begin{columns}
      \column{0.5\textwidth}

      \begin{eqnarray*}
        T\left(\vec{x}_1\right) & = & \hat{e}_1, \\
        T\left(\vec{x}_2\right) & = & \hat{e}_2, \\
        T\left(\vec{x}_3\right) & = & \hat{e}_3, \\
        \vdots & & \\
        T\left(\vec{x}_n\right) & = & \hat{e}_n, \\
      \end{eqnarray*}

      \column{0.5\textwidth}

      \only<2->{
        \begin{eqnarray*}
          S\left(\hat{e}_1\right) & = & \vec{x}_1, \\
          S\left(\hat{e}_2\right) & = & \vec{x}_2, \\
          S\left(\hat{e}_3\right) & = & \vec{x}_3, \\
          \vdots & & \\
          S\left(\hat{e}_n\right) & = & \vec{x}_n, \\
        \end{eqnarray*}
      }

      
    \end{columns}
  \end{frame}

  \begin{frame}{Using the RREF To Solve A System}

    \only<1>{
      \begin{eqnarray*}
        \left[
        \begin{array}{rrrrr|r}
          a_{11} & a_{12} & a_{13} & \cdots & a_{1n} & 1 \\
          a_{21} & a_{22} & a_{23} & \cdots & a_{2n} & 0 \\
          a_{31} & a_{32} & a_{33} & \cdots & a_{3n} & 0 \\
          \vdots & \vdots & \vdots & & \vdots \\
          a_{n1} & a_{n2} & a_{n3} & \cdots & a_{nn} & 0 \\
        \end{array}
        \right]
      \end{eqnarray*}
    }

    \only<2>{
      \begin{eqnarray*}
        \left[
        \begin{array}{rrrrr|r}
          1 & 0 & 0 & \cdots & 0 & x_{11} \\
          0 & 1 & 0 & \cdots & 0 & x_{21} \\
          0 & 0 & 1 & \cdots & 0 & x_{31} \\
          \vdots & \vdots & \vdots & & \vdots \\
          0 & 0 & 0 & \cdots & 1 & x_{n1} \\
        \end{array}
        \right]
      \end{eqnarray*}
    }

    
  \end{frame}

  \begin{frame}{Using the RREF To Solve A System}

    \only<1>{
      \begin{eqnarray*}
        \left[
        \begin{array}{rrrrr|r}
          a_{11} & a_{12} & a_{13} & \cdots & a_{1n} & 0 \\
          a_{21} & a_{22} & a_{23} & \cdots & a_{2n} & 1 \\
          a_{31} & a_{32} & a_{33} & \cdots & a_{3n} & 0 \\
          \vdots & \vdots & \vdots & & \vdots \\
          a_{n1} & a_{n2} & a_{n3} & \cdots & a_{nn} & 0 \\
        \end{array}
        \right]
      \end{eqnarray*}
    }

    \only<2>{
      \begin{eqnarray*}
        \left[
        \begin{array}{rrrrr|r}
          1 & 0 & 0 & \cdots & 0 & x_{12} \\
          0 & 1 & 0 & \cdots & 0 & x_{22} \\
          0 & 0 & 1 & \cdots & 0 & x_{32} \\
          \vdots & \vdots & \vdots & & \vdots \\
          0 & 0 & 0 & \cdots & 1 & x_{n2} \\
        \end{array}
        \right]
      \end{eqnarray*}
    }

    
  \end{frame}

    \begin{frame}{Using the RREF To Solve A System}

    \only<1>{
      \begin{eqnarray*}
        \left[
        \begin{array}{rrrrr|rrrrr}
          a_{11} & a_{12} & a_{13} & \cdots & a_{1n} & 1 & 0 & 0 & \cdots & 0 \\
          a_{21} & a_{22} & a_{23} & \cdots & a_{2n} & 0 & 1 & 0 & \cdots & 0 \\
          a_{31} & a_{32} & a_{33} & \cdots & a_{3n} & 0 & 0 & 1 & \cdots & 0 \\
          \vdots & \vdots & \vdots & & \vdots & \vdots & & & & \vdots \\
          a_{n1} & a_{n2} & a_{n3} & \cdots & a_{nn} & 0 & 0 & 0 & \cdots & 1 \\
        \end{array}
        \right]
      \end{eqnarray*}
    }

    \only<2>{
      \begin{eqnarray*}
        \left[
        \begin{array}{rrrrr|rrrrr}
          1 & 0 & 0 & \cdots & 0 & x_{11} & x_{12} & x_{13} & \cdots & x_{1n} \\
          0 & 1 & 0 & \cdots & 0 & x_{21} & x_{22} & x_{23} & \cdots & x_{2n} \\
          0 & 0 & 1 & \cdots & 0 & x_{31} & x_{32} & x_{33} & \cdots & x_{3n} \\
          \vdots & \vdots & \vdots & & \vdots & \vdots & & & & \vdots \\
          0 & 0 & 0 & \cdots & 1 & x_{n1} & x_{n2} & x_{n3} & \cdots & x_{nn} \\
        \end{array}
        \right]
      \end{eqnarray*}
    }

    
  \end{frame}


  \begin{frame}{Determine The Inverse}

    \begin{eqnarray*}
      \twoByTwo{1}{2}{1}{3}
    \end{eqnarray*}

    \dotfield{60}{22}
    
  \end{frame}

\begin{frame}{Blank Page}
  \dotfield{60}{24}
\end{frame}


\end{document}
